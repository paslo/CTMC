\documentclass[11pt]{article}
\title{UTOPIAE Book of Lectures:\\Markov Chain Chapter -- Table of Contents Draft}
\date{}
\author{}

\usepackage[inner=2.5cm,outer=2.5cm,top=3cm,bottom=3cm]{geometry}

\begin{document}
\maketitle

\vspace{-55pt}

\section{Introduction stochastic processes}
$\Rightarrow$ discrete + continuous time
\vspace{-12pt}
\section{Some notation \& first definitions}
$\Rightarrow$ stochastic processes as distributions on paths (not too formal, skip all the measure theory)
\vspace{-12pt}
\section{Discrete-Time Markov Chains}

\subsection{Markov Chains \& parameterisation}
\subsection{Markov Chains as event trees, further independence concepts}
\subsection{Markov Chains as probabilistic graphical models}
\subsection{Computations \& law of iterated expectation}

\section{Continuous-Time Markov Chains}
\subsection{Continuous-time as limit of discrete-time}
\subsection{Parameterisation \& the matrix exponential}
\subsection{Restrictions give graphical models}
CTMC $\Rightarrow$ non-homogeneous discrete-time MC $\Rightarrow$ computations follow immediately.
\vspace{-12pt}
\section{Imprecise stochastic processes}
\subsection{Parameterisation \& imprecise local models}
\subsection{Imprecise graphical models (credal networks)}
$\Rightarrow$ discrete-time Markov chains (** credality of local transition models is by assumption)

\noindent $\Rightarrow$ continuous-time Markov chains (** need to prove that this isn't an outer approximation)
\vspace{-12pt}
\subsection{Notes on independence concepts}
\subsection{Computations using law of iterated lower expectation}
$\Rightarrow$  just different local computations for discrete/continuous-time.
\vspace{-12pt}
\section{Summary \& further reading}

\end{document}
