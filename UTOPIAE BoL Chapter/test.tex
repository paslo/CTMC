\documentclass[11pt]{article}
\title{UTOPIAE Book of Lectures:\\Markov Chain Chapter -- Table of Content}
\date{}
\author{}

\usepackage[inner=2.5cm,outer=2.5cm,top=3cm,bottom=3cm]{geometry}

\begin{document}
\maketitle

\vspace{-55pt}

%\section{Introduction stochastic processes}
%$\Rightarrow$ discrete + continuous time
%\vspace{-12pt}
%\section{Some notation \& first definitions}
%$\Rightarrow$ stochastic processes as distributions on paths (not too formal, skip all the measure theory)
%\vspace{-12pt}
%\section{Discrete-Time Markov Chains}
%
%\subsection{Markov Chains \& parameterisation}
%\subsection{Markov Chains as event trees, further independence concepts}
%\subsection{Markov Chains as probabilistic graphical models}
%\subsection{Computations \& law of iterated expectation}
%
%\section{Continuous-Time Markov Chains}
%\subsection{Continuous-time as limit of discrete-time}
%\subsection{Parameterisation \& the matrix exponential}
%\subsection{Restrictions give graphical models}
%CTMC $\Rightarrow$ non-homogeneous discrete-time MC $\Rightarrow$ computations follow immediately.
%\vspace{-12pt}
%\section{Imprecise stochastic processes}
%\subsection{Parameterisation \& imprecise local models}
%\subsection{Imprecise graphical models (credal networks)}
%$\Rightarrow$ discrete-time Markov chains (** credality of local transition models is by assumption)
%
%\noindent $\Rightarrow$ continuous-time Markov chains (** need to prove that this isn't an outer approximation)
%\vspace{-12pt}
%\subsection{Notes on independence concepts}
%\subsection{Computations using law of iterated lower expectation}
%$\Rightarrow$  just different local computations for discrete/continuous-time.
%\vspace{-12pt}
%\section{Summary \& further reading}

\section{Introduction}
Both discrete- and continuous-time stochastic processes, some intuition, areas of application, toward the imprecise case, motivation etc.

\section{(Precise) Stochastic Processes}
Some notation and formalisation. Different representations; probability measures (intuitively, not going to do the full measure theoretic treatment), event trees, graphical models (again: short, not the full general theory). Various independence concepts, and Markov chains as a special case. Computations using law of iterated expectation. Focus largely on discrete-time, with some first concepts about continuous-time. A brief discussion about limit behaviour.

\section{Imprecise Discrete-Time Markov Chains}
Parameterisation and imprecise local models. Different independence concepts. Different representations; sets of precise processes, (imprecise) event trees, imprecise graphical models. Computation using law of iterated lower expectation. Results on limit behaviour.

\section{Imprecise Continuous-Time Markov Chains}
Intuition as limit of discrete-time (imprecise) Markov chains. Paramerisation using local models and the (imprecise) matrix exponential. Recollection and translation of the different independence concepts. Different representations; sets of precise processes and imprecise graphical models. Reduction of computations to discrete-time case, and law of iterated lower expectation. Results on limit behaviour.

\section{Further Reading}
Brief summary of main points including pointers to further reading. Also references to related work that we did not cover, e.g. the martingale-theoretic approach.

\end{document}
