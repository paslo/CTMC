\documentclass{article}
\usepackage{comment}
\usepackage{amssymb,amsthm}
\usepackage{algorithmic}
\usepackage{algorithm}
\usepackage{mathtools}
\usepackage{tikz}
  \usetikzlibrary{calc,matrix,plotmarks,trees,arrows,automata}


% sets of numbers
\newcommand{\nats}{\mathbb{N}}
\newcommand{\natswithzero}{\mathbb{N}_0}
\newcommand{\reals}{\mathbb{R}}

\newcommand{\statesymbol}{X}
\newcommand{\statessymbol}{\mathcal{X}}
\newcommand{\matrices}{\mathcal{Q}}
\newcommand{\lmatrixi}[1]{\underline{\matrices}_{#1}}
\newcommand{\umatrixi}[1]{\overline{\matrices}_{#1}}
\newcommand{\states}[1]{\statessymbol_{#1}}
\newcommand{\stateset}{\statessymbol}
\newcommand{\lexp}{\underline{E}}
\newcommand{\uexp}{\overline{E}}
\newcommand{\norm}[1]{\left\lVert #1 \right\rVert}
\newcommand{\abs}[1]{\left\vert #1 \right\vert}
\newcommand{\de}{\delta}
\newcommand{\ve}{\varepsilon}

\newtheorem{Theorem}{Theorem}
\newtheorem{Definition}{Definition}
\newtheorem{Lemma}{Lemma}
\newtheorem{Corollary}{Corollary}
\newtheorem{Proposition}{Proposition}
\newtheorem{Assumption}{Assumption}

\title{}
\date{}
\author{
  % Stavros Lopatatzidis \and Jasper De Bock \and Gert de Cooman\\
  % Ghent University, SYSTeMS Research Group\\
   % \{Stavros.Lopatatzidis,Jasper.DeBock,Gert.deCooman\}@UGent.be
}

\begin{document}
\maketitle
\section{Preiliminaries}

\subsection{Markov Chains and Expectations}
\textcolor{red}{*** This part is to be done later...}\\
We know from (precise) probability theory that given a function $f$ on the state space, the expectation is defined by 
\begin{equation*}
E
\end{equation*} 
Consider now a set of probability distributions over $\stateset$. 
Then, we call \emph{lower expectation} of $f$, the minimum expectation on $f$ that is obtained by any probability distribution in the set ...
That is 
\begin{equation*}
\lexp
\end{equation*} 
similarly, we have the \emph{upper expectation}, which is defined by....
\begin{equation*}
\uexp.
\end{equation*}  
One of the fundamental textbooks on imprecise probabilities and also on lower and upper expectations--often called as previsions--is the one in ref?.
\textcolor{red}{***}\\

\subsection{Norms}

The norm we consider for matrices is the infinite norm $\|\cdot\|_{\infty}$, defines by
\begin{equation*}
\norm{Q}_{\infty}\coloneqq \sup_{i\in\stateset}|\sum_{j\in\stateset}{Q(i,j)}|
\end{equation*}
with $Q$ an $|\stateset|\times|\stateset|$ matrix.\\

more to be added...
% Before presenting some properties around the expectations of a given vector $h$, we need to state some assumptions for the norms that we consider through out this document.

% \textcolor{red}{for matrices, we either have the operator norm or the $\infty$-norm}, for the vectors we use the supremum norm.
% The properties of the norms used are the following

\subsection{Imprecision}

Consider a continuous time Markov process with state space $\stateset\coloneqq\{0,\ldots,N\}$ and $N\in\nats$.\footnote{we do not consider zero to be a natural number.}
Such a process is characterised--in the precise case--by a single transition rate matrix $N\times N$ denoted by $Q$ and its of its elements by $Q(i,j)$ with $i,j\in\stateset$.
Each row $i$ of $Q$ represents the rates to move from state $i$ to any state of the process.
For any matrix $Q$, the following properties hold
\begin{itemize}
\item[P1.]$Q(i,j)\geq0$, $\forall i,j\in\stateset$ such that $i\neq j$
\item[P2.]$\sum_{j\in\stateset}Q(i,j)=0$, $\forall i\in\stateset$
\end{itemize}
Assume that instead of a single rate matrix, we are given a set of matrices $\matrices$.
Consider that for each matrix $Q$ in $\matrices$, we can determine the rates for each row separately.
That is, $\matrices_{i}\coloneqq\{Q(i,\cdot):Q\in\matrices\}$, for all $i\in\stateset$.
We assume that our matrices in $\matrices$ are those with rows determined separately.
\begin{Assumption} \label{ass:rates}
$Q\in\matrices$ if and only if $(\forall i\in\stateset) \ Q(i,\cdot)\in\matrices_{i}$
\end{Assumption}
We introduce some definitions for comparing vectors, which are based on point-wise comparison.
\begin{Definition} \label{def:leq_qh}
Given any real-valued function $h$ on $\stateset$ and $ Q, Q' \in \matrices$, then 
\begin{equation*}
\begin{split}
& Qh\leq  Q'h \text{ if and only if }  (Qh)(i)\leq  Q'h(i) \text{, for all } i\in\stateset
\text{ and }\\
& Qh\geq  Q'h \text{ if and only if }  (Qh)(i)\geq  Q'h(i) \text{, for all } i\in\stateset.
\end{split}
\end{equation*}
\end{Definition}

\begin{Definition} \label{def:l_qh}
Given any real-valued function $h$ on $\stateset$ and $ Q, Q' \in \matrices$, then 
\begin{equation*} 
\begin{split}
 Q h< Q'h &\text{ if and only if } \exists I\subseteq\stateset\text{ and }I\neq\varnothing \text{, such that }\\ 
& (Qh)(i')<  (Q'h)(i') \text{, for all } i'\in I \text{ and }\\
 & (Qh)(i)=  (Q'h)(i) \text{, for all } i\in\stateset\setminus I,
\end{split}
\end{equation*}
and similarly 
\begin{equation*} 
\begin{split}
 Qh> Q'h &\text{ if and only if } \exists I\subseteq\stateset\text{ and }I\neq\varnothing \text{, such that }\\ 
& (Q h)(i')>  (Q'h)(i') \text{, for all } i'\in I \text{ and }\\
& (Q h)(i)=  (Q'h)(i) \text{, for all } i\in\stateset\setminus I.
\end{split}
\end{equation*}
\end{Definition}
\begin{Definition} \label{def:eq_qh}
Given any real-valued function $h$ on $\stateset$ and $ Q, Q' \in \matrices$, then 
\begin{equation*}
 Q h= Q'h \text{ if and only if }  (Q h)(i)=  (Q'h)(i) \text{, for all } i\in\stateset.
\end{equation*}
\end{Definition}

\noindent
Based on Definitions \ref{def:leq_qh}, \ref{def:l_qh} and \ref{def:eq_qh}, we define the sets of matrices 
\begin{equation} \label{eq:lset1}
\lmatrixi{1}\coloneqq\left\{Q\in\matrices:Qh\leq Q'h\text{, }\forall Q'\in\matrices\setminus Q\right\} 
\end{equation} 
\begin{equation*}
\umatrixi{1}\coloneqq\left\{Q\in\matrices:Qh\geq Q'h\text{, }\forall Q'\in\matrices\setminus Q\right\}
\end{equation*} 

\noindent
From the  aforementioned definitions, we can derive the following corollaries
\begin{Corollary} \label{cor:comparison}
Given  $ Q, Q'$ in $\matrices$ and a real-valued function $h$ on $\stateset$, then  
\begin{equation*}
\begin{split}
& Q h\leq  Q'h \text{ if and only if }  Q h <  Q'h \text{ or }  Q h =  Q'h
\text{, and }\\
& Q h\geq  Q'h \text{ if and only if }  Q h >  Q'h \text{ or }  Q h =  Q'h.
\end{split}
\end{equation*}
\end{Corollary}
\begin{Corollary} \label{cor:empty}
The sets of matrices $\lmatrixi{1}$ and $\umatrixi{1}$ cannot be empty.
\end{Corollary}
\begin{proof}
Immediate from the Assumption \ref{ass:rates}.
\end{proof}

\noindent
We will focus on the set $\lmatrixi{1}$ and on the lower expectations and the results for the upper expectations follow similarly.


\begin{Proposition} \label{prop:singleton}
Given a real-valued function $h$ on $\stateset$, then $\lmatrixi{1}$ is a singleton if and only if it exists $ Q^{*}$ in $\matrices$, such that for all $ Q ' \in\matrices\setminus Q^{*}$
\begin{equation*}
 Q^{*}h< Q'h
\end{equation*} 
\end{Proposition}
\begin{proof}
First we prove that 
\begin{equation} \label{imp:1}
\exists Q^{*}\in\matrices\text{, such that } Q^{*}h< Q'h\text{, }\forall Q ' \in\matrices\setminus Q^{*}\Rightarrow \lmatrixi{1}=\{ Q^{*}\}
\end{equation}

\noindent
Since $ Q^{*}h< Q'h$ holds, it also holds---due to Corollary \ref{cor:comparison}---that $ Q^{*}h \leq Q'h$, for all $ Q ' \in\matrices\setminus Q^{*}$.
Therefore, due to Equation \eqref{eq:lset1}, $ Q^{*}\in\lmatrixi{1}$.
From the left side of the implication in \eqref{imp:1}, combined with Corollary \ref{cor:comparison}, we have that 
\begin{equation} \label{imp:2}
\begin{split}
&\exists Q^{*}\in\matrices\text{, such that } Q^{*}h< Q'h\text{, }\forall Q ' \in\matrices\setminus Q^{*}\Rightarrow \\
&\not\exists Q'\in\matrices\setminus Q^{*}\text{, such that } Q'h\leq Q^{*}h
\end{split}
\end{equation}
and because of Equation \eqref{eq:lset1} and Corollary \ref{cor:empty}, we have that $\lmatrixi{1}=\{ Q^{*}\}$.
% Assume that $\lmatrixi{1}$ does not contain only $ Q^{*}$.
% We will prove by contradiction that this statement is false.

% \noindent
% Let $ Q$ in $\lmatrixi{1}$ with $ Q\neq Q^{*}$.
% From Equation \eqref{eq:lset1}, we have that
% \begin{equation*}
%  Q h \leq  Q''h\text{, }\forall Q''\in\matrices\setminus Q,
% \end{equation*}
% which implies that
% \begin{equation} \label{ineq:leq}
%  Q h \leq  Q^{*}h.
% \end{equation}
% By taking the initial statement in \eqref{imp:1}, we have that
% \begin{equation*}
%  Q^{*} h <  Q'h\text{, }\forall Q'\in\matrices\setminus Q' 
% \end{equation*}
% which implies that
% \begin{equation*}
%  Q^{*} h <  Q h 
% \end{equation*}
% and which contradicts with Inequality \eqref{ineq:leq}.
% Hence, there is no $ Q$ in $\lmatrixi{1}$ with $ Q\neq Q^{*}$, and therefore $\lmatrixi{1}=\{ Q^{*}\}$.

\noindent
To complete the proof, we need to prove that
\begin{equation*} 
\lmatrixi{1}\text{ is a singleton }\Rightarrow \exists Q^{*}\in\matrices\text{, such that } Q^{*}h< Q'h\text{, }\forall Q ' \in\matrices\setminus Q^{*}
\end{equation*}
If $\lmatrixi{1}$ is a singleton, then, due to \eqref{eq:lset1}, we have that
\begin{equation*} 
\exists Q^{*}\in\matrices\text{, such that } Q^{*}h\leq Q'h\text{, }\forall Q ' \in\matrices\setminus Q^{*}
\end{equation*}
which, due to Corollary \ref{cor:comparison}, implies that
\begin{equation} \label{imp:3}
 Q^{*}h< Q'h \text{ or }  Q^{*}h= Q'h\text{, }\forall Q' \in\matrices\setminus Q^{*}. 
\end{equation}
Clearly, we cannot have $ Q^{*}h= Q'h$ as this would imply that $\lmatrixi{1}$ is not a singleton. 
Therefore, from \eqref{imp:3}, we conclude that $ Q^{*}h< Q'h \text{, }\forall Q' \in\matrices\setminus Q^{*}$.
\end{proof}

\noindent
Another property derived from the definitions is the following
\begin{Proposition} \label{prop:not_singleton}
Given a real-valued function $h$ on $\stateset$, then $\lmatrixi{1}$ is not a singleton if and only if for any $ Q', Q''$ in $\lmatrixi{1}$, it holds that 
\begin{equation*}
 Q'h= Q''h.
\end{equation*} 
\end{Proposition}
\begin{proof}
We prove the 'if' part as the 'only if' is trivial. 
\textcolor{red}{still need to be done, but it is only a few lines}
Given that $\lmatrixi{1}$, then (we have at least two matrices the one less or equal to the other, hence both of them equal...)since the matrices where chosen arbitrarily then it follows...
\end{proof}

Now we deal with the exponential of a given rate matrix $Q$, which is defined by the Taylor series as follows
\begin{equation}
e^{tQ}=\sum_{k=0}^{\infty}{\frac{t^{k}Q^{k}}{k!}}
\end{equation}
Our goal is to find minimum (or maximum), point-wise, expected values of $e^{tQ}h$ for a given function $h$ on $\stateset$ and $Q\in\matrices$.
We start with the following property, which relates the values of the exponentials $e^{Qt}h$ to the values of their first derivative with respect to time.
\begin{Proposition} \label{prop:exp_1}
Given a real-valued function $h$ on $\stateset$, and $ Q, Q'$ in $\matrices$, such that $ (Q h)(i)< (Q'h)(i)$ for all $i\in\stateset$.
Then, we have that
\begin{equation*}
(\exists\de>0)(\forall t\in(0,\de))\text{ }e^{t Q}h<e^{t Q'}h.
\end{equation*}
\end{Proposition}

\begin{proof}
% \begin{equation*}
% \mathbf{A}(t)=\frac{\partial}{\partial s}\bigl[\mathbf{T}_{t}^{s}\bigr]\Big\vert_{s=t}\,,
% \end{equation*}
We know that our system satisfies the following equation
\begin{equation} \label{eq:kol_forw}
\frac{\partial}{\partial t}\bigl[e^{tQ}h\bigr]\Big\vert_{t=0}=Q h.
\end{equation}
which comes from the definition of the first derivative of $e^{t Q}h$ evaluated at $t=0$.\footnote{we do not need the whole derivation of the equation because it already proved in some literature. We do so for ease of comprehension.}
That is 
\begin{equation*}
\begin{split}
&\lim_{c\to0}{\frac{e^{c Q}h-h}{c}}=\lim_{c\to0}{\frac{e^{c Q}-I}{c}h}=\lim_{c\to0}{\frac{\sum_{k=0}^{\infty}{\frac{c^{k}Q^{k}}{k!}}-I}{c}h}\\
=&\lim_{c\to0}{\frac{I+c Q+\sum_{k=2}^{\infty}{\frac{c^{k}Q^{k}}{k!}}-I}{c}h}=Q h+\lim_{c\to0}{\frac{\sum_{k=2}^{\infty}{\frac{c^{k}Q^{k}}{k!}}}{c}h}
\end{split}
\end{equation*}
where $I$ is the identity matrix.\\\\
Further on, we prove that 
\begin{equation*}
\lim_{c\to0}{\frac{\sum_{k=2}^{\infty}{\frac{c^{k}Q^{k}}{k!}}}{c}h}=0,
\end{equation*}
which is true, if the following holds
\begin{equation*}
(\forall\ve>0)(\exists \de>0)(\forall t \in(0,\de))\norm{\frac{1}{t}\sum_{k=2}^{\infty}{\frac{t^{k}Q^{k}}{k!}}h}<\ve.
\end{equation*}
For the aforementioned norm, we have that
\begin{equation*}
\begin{split}
&\norm{\frac{1}{t}\sum_{k=2}^{\infty}{\frac{t^{k}Q^{k}}{k!}}h}=\frac{1}{t}\norm{\sum_{k=2}^{\infty}{\frac{t^{k}Q^{k}}{k!}}h}\leq\frac{1}{t}\sum_{k=2}^{\infty}{\norm{\frac{t^{k}Q^{k}}{k!}h}}=\frac{1}{t}\sum_{k=2}^{\infty}{\frac{t^{k}}{k!}\norm{Q^{k}h}}\\
\leq&\frac{1}{t}\sum_{k=2}^{\infty}{\frac{t^{k}}{k!}\norm{Q^{k}}\norm{h}}\leq\frac{1}{t}\sum_{k=2}^{\infty}{\frac{t^{k}}{k!}\norm{Q}^{k}\norm{h}}=\frac{1}{t}(e^{t\norm{Q}}-1-t\norm{Q})\norm{h}.
\end{split}
\end{equation*}
where all the steps, except for the last, come from the properties of the norm used.
The last step comes from the Taylor series, which is used to calculate an exponential as follows
\begin{equation} \label{taylor}
e^{t\alpha}=\sum_{k=0}^{\infty}\frac{t^{k}\alpha^{k}}{k!}
\end{equation}
for any $\alpha\in\reals$.
In order to complete the proof of $\lim_{c\to0}{\frac{\sum_{k=2}^{\infty}{\frac{c^{k}Q^{k}}{k!}}}{c}h}=0$, we need to prove that
\begin{equation} \label{eq:lim0}
(\forall\ve>0)(\exists \de'>0)(\forall t \in(0,\de'))\norm{\frac{1}{t}\sum_{k=2}^{\infty}{\frac{t^{k}Q^{k}}{k!}}h}\leq\frac{1}{t}(e^{t\norm{Q}}-1-t\norm{Q})\norm{h}<\ve,
\end{equation}
for which we have
\begin{equation*}
\frac{1}{t}(e^{t\norm{Q}}-1-t\norm{Q})\norm{h}<\ve\Rightarrow \frac{e^{t\norm{Q}}-1}{t}-\norm{Q}<\frac{\ve}{\norm{h}}.
\end{equation*}
By setting $\ve'=\frac{\ve}{\norm{h}}$, we have that
\begin{equation*}
\frac{e^{t\norm{Q}}-1}{t}-\norm{Q}<\ve'\Rightarrow \abs{\frac{e^{t\norm{Q}}-1}{t}-\norm{Q}}<\ve'
\end{equation*}
where the implication comes from \eqref{eq:lim0} and it further implies that
\begin{equation*}
(\forall\ve'>0)(\exists \de'>0)(\forall t \in(0,\de'))\abs{\frac{e^{t\norm{Q}}-1}{t}-\norm{Q}}<\ve'.
\end{equation*}
Therefore, we have that
\begin{equation*}
(\forall\ve'>0)(\exists \de'>0)(\forall t \in(0,\de'))\norm{\frac{1}{t}\sum_{k=2}^{\infty}{\frac{t^{k}Q^{k}}{k!}}h}<\ve'
\end{equation*}

\noindent
\noindent
To continue with the main proof, we take again Equation~\eqref{eq:kol_forw}, we have that
\begin{equation*}
(\forall\ve>0)(\exists \de'>0)(\forall t \in(0,\de'))\norm{\frac{e^{tQ}-I}{t}h-Qh}<\ve
\end{equation*}
which implies that
\begin{equation*}
t\norm{\frac{e^{tQ}-I}{t}h-Qh}<t\ve\Rightarrow \norm{e^{tQ}h-h-tQh}<t\ve
\end{equation*}
which holds still for any $\ve>0$, by taking $\de^{*}\coloneqq\min\{\de',1\}$, which implies that $t\ve<\de^{*}\ve\leq\ve$.
Hence, for any $Q,Q'\in \matrices$, it holds that
\begin{equation*}
\begin{split}
&(\forall\ve>0)(\exists \de>0)(\forall t \in(0,\de)) \norm{e^{tQ}h-h-tQh}<t\ve\\
&(\forall\ve>0)(\exists \de'>0)(\forall t \in(0,\de')) \norm{e^{tQ'}h-h-tQ'h}<t\ve
\end{split}
\end{equation*}
Now consider $\de_{1}\coloneqq\{\de,\de'\}$, then
\begin{equation*}
\begin{split}
&(\forall\ve>0)(\forall t \in(0,\de_{1})) \norm{e^{tQ}h-h-tQh}<t\ve\\
&(\forall\ve>0)(\forall t \in(0,\de_{1})) \norm{e^{tQ'}h-h-tQ'h}<t\ve
\end{split}
\end{equation*}
Taking the second statement, we have that
\begin{equation*}
\norm{e^{tQ'}h-h-tQ'h}<t\ve \Rightarrow \sup_{i\in\stateset}|(e^{tQ'}h-h-tQ'h)(i)|<t\ve
\end{equation*}
which implies that
\begin{equation} \label{eq:exp_abs}
|(e^{tQ'}h-h-tQ'h)(i)|<t\ve \Rightarrow -t\ve<(e^{tQ'}h-h-tQ'h)(i)<t\ve \text{, for all } i\in\stateset.
\end{equation}
similar arguments hold for $Q$ as well.
Taking \eqref{eq:exp_abs}, we have that
\begin{equation} \label{eq:exp_geq}
(e^{tQ'}h)(i) > h(i)+t(Q'h)(i)-t\ve 
\end{equation}
Since we know that $(Q'h)(i)>(Qh)(i)$ for all $i\in\stateset$, then we have that
\begin{equation} \label{eq:beta}
(\exists b>0) Q'h)(i)>(Qh)(i)+b \text{ for all } i\in\stateset
\end{equation} 
Taking inequality \eqref{eq:exp_geq}, we have that
\begin{equation} \label{eq:last}
\begin{split}
&(e^{tQ'}h)(i) > h(i)+t(Q'h)(i)-t\ve > h(i)+t(Qh)(i)+tb-t\ve \\
>& (e^{tQ'}h)(i)-t\ve+tb-t\ve=(e^{tQ}h)(i)+t(b-2\ve),
\end{split}
\end{equation}
where the second inequality comes from \eqref{eq:beta} and the third from \eqref{eq:exp_geq} with respect to $Q$.
Take now inequality \eqref{eq:last} for $\ve=c/2$, we have $e^{tQ'}h>e^{tQ}$ for any $t\in(0,\de^{*})$.
\end{proof}
\textcolor{red}{I am not exactly sure how it is proven for the case when $Q'h\geq Qh$, where for some $i$ it is $Q'h(i)=Qh(i)$ and for the rest $i'$ we have $Q'h(i')>Qh(i')$}

It is possible that we have $Q'h=Qh$, for $Q'\neq Q$.
The following proposition shows that we can use the second derivative of $e^{tQ}h$ and $e^{tQ'}h$ in order to compare them point-wise.
% An example of that case is the probability of state $1$, for state space $\stateset=\{0,1,2,3,4\}$, which is the indicator function
\begin{Proposition} \label{prop:exp_2}
Given a real-valued function $h$ on $\stateset$, and $ Q, Q'$ in $\matrices$, such that $Q h=Q'h$ and $(Q'^{2}h)(i)>(Q^{2}h)(i)$ for all $i\in\stateset$.
Then, we have that
\begin{equation*}
(\exists\de>0)(\forall t\in(0,\de))\text{ }e^{t Q}h<e^{t Q'}h.
\end{equation*}
\end{Proposition}

\begin{proof}
% \begin{equation*}
% \mathbf{A}(t)=\frac{\partial}{\partial s}\bigl[\mathbf{T}_{t}^{s}\bigr]\Big\vert_{s=t}\,,
% \end{equation*}
We know that our system satisfies the following equation
\begin{equation} \label{eq:kol_forw_2}
\frac{\partial^{2}}{\partial^{2} t}\bigl[e^{tQ}h\bigr]\Big\vert_{t=0}=Q^{2} h.
\end{equation}
which comes from the definition of the second derivative of $e^{t Q}h$ evaluated at $t=0$.
That is 
\begin{equation*}
\begin{split}
&\lim_{c\to0}\frac{\frac{\partial}{\partial t}[e^{(t+c)Q}]\Big\vert_{t=0}-\frac{\partial}{\partial t}[e^{tQ}]\Big\vert_{t=0}}{c}h=
\lim_{c\to0}\frac{\frac{e^{2c}-e^{c}}{c}-\frac{e^{2c}-I}{c}}{c}h=\\
=&\lim_{c\to0}{\frac{e^{2cQ}-2e^{cQ}+I}{c^{2}}h}=\lim_{c\to0}\frac{(e^{cQ}-I)^{2}}{c^{2}}h=(\lim_{c\to0}\frac{e^{cQ}-I}{c})^{2}h=Q^{2}h
\end{split}
\end{equation*}
where $I$ is the identity matrix.\\

\noindent
By taking Equation~\eqref{eq:kol_forw_2}, we have that
\begin{equation*}
(\forall\ve>0)(\exists \de>0)(\forall t \in(0,\de))\norm{\frac{(e^{tQ}-I)^{2}}{t^{2}}h-Q^{2}h}<\ve
\end{equation*}
which implies that
\begin{equation*}
t^{2}\norm{\frac{(e^{tQ}-I)^{2}}{t^{2}}h-Q^{2}h}<t^{2}\ve\Rightarrow \norm{(e^{tQ}-I)^{2}h-t^{2}Q^{2}h}<t^{2}\ve
\end{equation*}
which holds still for any $\ve>0$, by taking $\de^{*}\coloneqq\min\{\de,1\}$, which implies that $t^{2}\ve<\de^{*}\ve\leq\ve$.
Hence, for $Q,Q'\in \matrices$, it holds that
\begin{equation*}
\begin{split}
&(\forall\ve>0)(\exists \de>0)(\forall t \in(0,\de)) \norm{(e^{tQ}-I)^{2}h-t^{2}Q^{2}h}<t^{2}\ve\\
&(\forall\ve>0)(\exists \de'>0)(\forall t \in(0,\de')) \norm{(e^{tQ'}-I)^{2}h-t^{2}Q'^{2}h}<t^{2}\ve
\end{split}
\end{equation*}
Now consider $\de_{2}\coloneqq\{\de,\de'\}$, then
\begin{equation*}
\begin{split}
&(\forall\ve>0)(\exists \de_{2}>0)(\forall t \in(0,\de_{2})) \norm{(e^{tQ}-I)^{2}h-t^{2}Q^{2}h}<t^{2}\ve\\
&(\forall\ve>0)(\exists \de'_{2}>0)(\forall t \in(0,\de'_{2})) \norm{(e^{tQ'}-I)^{2}h-t^{2}Q'^{2}h}<t^{2}\ve
\end{split}
\end{equation*}
Taking the second statement, we have that
\begin{equation*}
\norm{(e^{tQ'}-I)^{2}h-tQ'^{2}h}<t^{2}\ve \Rightarrow \sup_{i\in\stateset}|((e^{tQ'}-I)^{2}h-tQ'^{2}h)(i)|<t^{2}\ve
\end{equation*}
which implies that
\begin{equation} \label{eq:exp_abs_2}
|((e^{tQ'}-I)^{2}h-tQ'^{2}h)(i)|<t^{2}\ve \Rightarrow -t^{2}\ve<((e^{tQ'}-I)^{2}h-tQ'^{2}h)(i)<t^{2}\ve \text{, } \forall i\in\stateset.
\end{equation}
similar arguments hold for $Q$ as well.
Taking \eqref{eq:exp_abs_2}, we have that
\begin{equation} \label{eq:exp_geq_2}
((e^{tQ'}-I)^{2}h)(i) > h(i)+t^{2}(Q'^{2}h)(i)-t^{2}\ve 
\end{equation}
in which the left hand is equal to 
\begin{equation*}
((e^{tQ'}-I)^{2}h)(i)=(e^{2tQ'}h)(i)-2(e^{tQ'}h)(i)+h(i).
\end{equation*}
Now, Inequality \eqref{eq:exp_geq_2} becomes
\begin{equation} \label{eq:exp_geq_2b}
(e^{2tQ'}h)(i) > 2(e^{tQ'}h)(i)-h(i)+t^{2}(Q'^{2}h)(i)-t^{2}\ve 
\end{equation}
Since we know that $(Q'^{2}h)(i)>(Q^{2}h)(i)$ for all $i\in\stateset$, then we have that
\begin{equation} \label{eq:beta_2}
(\exists b>0) Q'^{2}h)(i)>(Q^{2}h)(i)+b \text{ for all } i\in\stateset
\end{equation} 
Taking inequality \eqref{eq:exp_geq_2b}, we have that
\begin{equation} \label{eq:exp_geq_3}
(e^{2tQ'}h)(i) > 2(e^{tQ'}h)(i)-h(i)+t^{2}(Q^{2}h)(i)+t^{2}b-t^{2}\ve 
\end{equation}
From inequality \eqref{eq:last} in Proposition \ref{prop:exp_1}, we know that
\begin{equation*} 
e^{Q't}h>e^{Qt}h+t(b-2\ve)\text{, with } b>0
\end{equation*}
and since now $Q'h=Qh$, which means that $b=0$, we have that $e^{Q't}h>e^{Qt}h-2t\ve$.
Combining this with Inequality \eqref{eq:exp_geq_3}, we have that
\begin{equation*} 
(e^{2tQ'}h)(i) > 2(e^{tQ}h)(i)-4t\ve-h(i)+t^{2}(Q^{2}h)(i)+t^{2}b-t^{2}\ve 
\end{equation*}
which combined with \eqref{eq:exp_abs_2}, with respect to $Q$ results in
\begin{equation} \label{eq:last_2}
(e^{2tQ'}h)(i) > (e^{2tQ}h)(i)-4t\ve+t^{2}b-2t^{2}\ve 
\end{equation}
and by selecting an $\ve$, such that $t^{2}b-2t^{2}\ve-4t\ve=0$, we have $e^{2tQ'}h>e^{2tQ}$ for any $t\in(0,\de^{*})$.\\\\
\end{proof}

***

Remianing....case equal or smaller....genral case for $k$th derivative.
The rest sections are the same in comments, nothing is else is added.

***

\begin{comment}

\section{General Description}

We introduce the notion of flipping times for imprecise continuous-time birth-death processes, show how to obtain them, and explain how they lead to new computational methods.
\paragraph{The Precise Case} 
Consider a continuous-time Markov processes where, at any time $t$, the stochastic matrix of the process $P_{t}$ is derived from a generator matrix~$Q$.
When $Q$ is bounded, $P_{t}$ satisfies the Kolmogorov backward equation

\begin{equation} \label{Kol_eq}
\frac{d}{dt}P_{t}=QP_{t}.
\end{equation}
If we let $f_{t}(x)\coloneqq E_{t}(f\vert X_0=x)$, with $f$ a real-valued function on the finite state space~$\mathcal{X}$ and $x\in\mathcal{X}$ an initial state, then
we can rewrite Equation~\eqref{Kol_eq} as follows:
\begin{equation} \label{T_Kol_eq}
\frac{d}{dt}f_{t}=Qf_{t}.
\vspace{3pt}
\end{equation}
Combined with the boundary condition $f_0=f$, the unique solution of Equation~\eqref{T_Kol_eq} is $f_{t}=e^{Qt}f$.

Instead of considering a time-invariant $Q$, we can also let $Q_{t}$ be a function of the time $t$.
 In that case, Equation \eqref{T_Kol_eq} can be rewritten as
 \begin{equation} \label{time_Kol_eq}
\frac{d}{dt}f_{t}=Q_{t}f_{t}. 
\end{equation}
In general, Equation~\eqref{time_Kol_eq} has no analytical solution.
\paragraph{The Imprecise Case}We focus on the case where every state in $\stateset\coloneqq \{0,\ldots,L\}$, %with $L \in\nats$.
has an interval-valued birth and/or death rate. The generator matrix is then a tridiagonal matrix of the form 

\begin{equation*}
 \begin{pmatrix}
  -\lambda_{0} & \lambda_{0} & 0 & \cdots & \cdots & \cdots & 0 \\
  \vdots & \ddots & \ddots & \ddots & \ddots & \ddots & \vdots  \\
  0 & \cdots & \mu_{i} & -(\mu_{i}+\lambda_{i}) & \lambda_{i} & \cdots & 0 \\
  \vdots & \ddots & \ddots & \ddots & \ddots & \ddots & \vdots  \\
  0 & \cdots & \cdots &\cdots & 0 & \mu_{L} & -\mu_{L}
 \end{pmatrix}
\end{equation*}
where, for all $i\in\{0,\dots,L-1\}$ and $j\in\{1,\dots,L\}$, $\lambda_{i} \in [\underline{\lambda},\overline{\lambda}]$ and $\mu_{j} \in [\underline{\mu},\overline{\mu}]$. \textbf{We use $\mathcal{Q}$ to denote the the set that consists of all these generator matrices}.

At any time $t$, the only assumption we make about $Q_{t}$ is that it is an element of $\mathcal{Q}$. Every such possible choice of non-stationary generator matrices will, by Equation~\eqref{time_Kol_eq}, result in a---possibly different---solution $f_t$. Our goal is to calculate exact lower and upper bounds for the set of all these solutions $f_{t}$, as denoted by $\smash{\underline{f}_{t}}$ and $\smash{\overline{f}_{t}}$; we focus on the lower bound here.
As proved by \v{S}kulj~\cite{skulj2015}, $\smash{\underline{f}_{t}}$ is the solution to
\begin{equation} \label{ip_Kol_eq}
\frac{d}{dt}\underline{f}_{t}=\min_{Q\in \mathcal{Q}}Q\underline{f}_{t},
\end{equation}
with boundary condition $\underline{f}_0=f$.
If $\mathcal{Q}$ is the convex hull of a \emph{finite} number of extreme generator matrices---as in our case---then since the solution to the above differential equation is continuous, we find that there must be time points $0=t_0<t_1<\ldots<t_{i}<t_{i+1}<\ldots$ such that, for all $t\in[t_i,t_{i+1}]$, the minimum in Equation~\eqref{ip_Kol_eq} is obtained by the same extreme generator matrix $Q_i\in\mathcal{Q}$. We call these time points $t_i$ \emph{flipping times}. The differential equation~\eqref{ip_Kol_eq} is then piecewise linear, and the solution is therefore given by
\begin{equation*}
\underline{f}_t=e^{Q_i(t-t_i)}e^{Q_{i-1}(t_i-t_{i-1})}\dots e^{Q_1(t_2-t_1)}e^{Q_0(t_1)}f,
\end{equation*}
for $t\in[t_i,t_{i+1}]$. The difficult part is now to find the flipping times $t_i$ and the corresponding extreme generator matrices $Q_i$. 
Below, we show the procedure that we follow in order to identify $t_i$ and $Q_i$, with respect to lower bounds.\\

\begin{algorithmic}[1]
\STATE \textbf{Input}($\stateset:=\{0,\ldots,N\},f:\stateset\to\mathbb{R},\mathcal{Q}$)
\STATE \ \ \  $i \leftarrow 0$
\STATE \ \ \  $t_{i} \leftarrow 0$
\STATE \ \ \  $f_{t_{i}} \leftarrow f$
\STATE \ \ \  $\underline{Q}_{i}\leftarrow \arg\min_{Q\in\mathcal{Q}}Qf_{t_{i}}$
\STATE \ \ \ \textbf{repeat}
  \STATE \ \ \ \ \ \ $t'$ := $\min_{Q\in\mathcal{Q}}\{t:Qe^{\underline{Q}_{i}t}f_{t_{i}}=\underline{Q}_{i}e^{\underline{Q}_{i}t}f_{t_{i}}\}$ 
  \STATE \ \ \ \ \ \ $Q'$ := $\arg\min_{Q\in\mathcal{Q}}\{t:Qe^{\underline{Q}_{i}t}f_{t_{i}}=\underline{Q}_{i}e^{\underline{Q}_{i}t}f_{t_{i}}\}$
  \STATE \ \ \ \ \ \ $i \leftarrow i+1$
  \STATE \ \ \ \ \ \ $t_{i} \leftarrow t'$ 
  \STATE \ \ \ \ \ \ $\underline{Q}_{i} \leftarrow Q'$
  \STATE \ \ \ \ \ \ $f_{t_{i}} \leftarrow e^{\underline{Q}_{i-1}t_{i}}f_{t_{i-1}}$ 
\STATE \ \ \ \textbf{until}  $t_i \not\in \mathbb{R}_{+}$ 
\STATE \ \ \ \textbf{return} $e^{\underline{Q}_{i}t}f_{t_{i}}$ for $t\to\infty$
\newline
\end{algorithmic}
In step 13, $\mathbb{R}_{+}$ is the set of all positive real numbers.
The minimum in steps 5 and 7 are taken row-wise.
We provide some examples to check how the method works.


\section{Examples}

We calculate lower and upper probabilities for state spaces of size 3 and 4.
For all the following examples, we consider a set of generator matrices $\mathcal{Q}$ derived from the intervals $\lambda \in[1,3]$ and $\mu \in [2,5]$.
The numerical values are presented up to their 5th decimal point, for ease of representation. 
The actual values are in comments in the .tex file.\\

\textbf{Lower and Upper probabilities for state space is $\stateset:=\{0,1,2\}$}\\

--- \textbf{State 0} ($f\coloneqq[1,0,0]^T$) ---\\

***Lower probability ($\underline{E}(X=0)$)***\\

First we have to find the generator matrix $Q$ that minimises $Qf$.
We see that the form of the vector $Qf$ is
\begin{equation*}  
 \begin{array}{|c|}
  \lambda_{0}(f(1)-f(0))\\
  \mu_{1}(f(0)-f(1))+\lambda_{1}(f(2)-f(1))\\
  \mu_{2}(f(1)-f(2))  
 \end{array}
 \end{equation*}
from which we have that $\lambda_{0}=\overline{\lambda}=3$ and $\mu_{1}=\underline{\mu}=2$.
We see that we cannot directly determine $\lambda_{1}$ and $\mu_{2}$ since $f(1)-f(2)=0$.
Hence, the form of $\underline{Q}_0$ is
\begin{equation*} 
 \underline{Q}_{0}= 
 \begin{array}{|rrr|}
  -3 & 3 & 0 \\
  2 & ? & ?  \\
  0 & ? & ?  
 \end{array}
 \end{equation*}
 where for the unknown values (?), we test the possible combinations of extreme points $\{\underline{\lambda},\overline{\lambda}\}\times\{\underline{\mu},\overline{\mu}\}$.
 Therefore, we have to check which of the following generator matrices
\begin{equation*}
\begin{split} 
 &Q_{0a}= 
 \begin{array}{|rrr|}
  -3 & 3 & 0 \\
  2 & -3 & 1 \\
  0 & 2 & -2  
 \end{array}\ \  
 Q_{0b}=
 \begin{array}{|rrr|}
  -3 & 3 & 0 \\
  2 & -3 & 1 \\
  0 & 5 & -5  
 \end{array}\\
 &Q_{0c}=
 \begin{array}{|rrr|}
  -3 & 3 & 0 \\
  2 & -5 & 3 \\
  0 & 2 & -2  
 \end{array}\ \ 
 Q_{0d}=
 \begin{array}{|rrr|}
  -3 & 3 & 0 \\
  2 & -5 & 3 \\
  0 & 5 & -5  
 \end{array}
 \end{split}
 \end{equation*}
 yields the minimum $f_{t_{1}}$, with $t_1$ the possible flipping time.\\\\

--- First Iteration ---\\
We start with the generator matrix 
\begin{equation*} 
 Q_{0a}= 
 \begin{array}{|rrr|}
  -3 & 3 & 0 \\
  2 & -3 & 1 \\
  0 & 2 & -2  
 \end{array}
 \end{equation*}
which is diagonalisable and yields the following results
\begin{equation*} 
 e^{Q_{0a}t}= 
 \begin{array}{|rrr|}
  0.30+0.41e^{-5.73t}+0.27e^{-2.26t} & 0.46-0.56e^{-5.73t}+0.10e^{-2.26t} & 0.23+0.15e^{-5.73t}-0.38e^{-2.26t} \\
  0.30-0.37e^{-5.73t}+0.06e^{-2.26t} & 0.46+0.51e^{-5.73t}+0.02e^{-2.26t} & 0.23-0.13e^{-5.73t}-0.09e^{-2.26t} \\
  0.30+0.20e^{-5.73t}-0.05e^{-2.26t} & 0.46-0.27e^{-5.73t}-0.18e^{-2.26t} & 0.23+0.07e^{-5.73t}+0.69e^{-2.26t}  
 \end{array}
 \end{equation*}
 %Matrix(3, 3, {(1, 1) = .3076923077+.4127711850*exp(-5.732050808*t)+.2795365072*exp(-2.267949192*t), (1, 2) = .102317463*exp(-2.267949192*t)-.5638559244*exp(-5.732050808*t)+.4615384613, (1, 3) = -.3818539704*exp(-2.267949192*t)+.1510847396*exp(-5.732050808*t)+.2307692308, (2, 1) = -.3759039497*exp(-5.732050808*t)+.3076923075+0.682116422e-1*exp(-2.267949192*t), (2, 2) = .4615384619+0.249671936e-1*exp(-2.267949192*t)+.5134943446*exp(-5.732050808*t), (2, 3) = -0.931788359e-1*exp(-2.267949192*t)-.1375903950*exp(-5.732050808*t)+.2307692309, (3, 1) = -.5091386272*exp(-2.267949192*t)+.2014463194*exp(-5.732050808*t)+.3076923078, (3, 2) = -.2751807900*exp(-5.732050808*t)+.4615384614-.1863576714*exp(-2.267949192*t), (3, 3) = .2307692306+.6954962990*exp(-2.267949192*t)+0.7373447050e-1*exp(-5.732050808*t)})
\begin{equation} \label{eQ0atf}
 e^{Q_{0a}t}f= 
 \begin{array}{|r|}
  0.30769+0.41277e^{-5.73205t}+0.27953e^{-2.26794t}\\
  0.30769-0.37590e^{-5.73205t}+0.68211e^{-2.26794t}\\
  0.30769+0.20144e^{-5.73205t}-0.50913e^{-2.26794t}
 \end{array}
 \end{equation}
%Matrix(3, 1, {(1, 1) = .3076923077+.4127711850*exp(-5.732050808*t)+.2795365072*exp(-2.267949192*t), (2, 1) = -.3759039497*exp(-5.732050808*t)+.3076923075+0.6821164220e-1*exp(-2.267949192*t), (3, 1) = -.5091386272*exp(-2.267949192*t)+.2014463194*exp(-5.732050808*t)+.3076923078})
In order to find the flipping times, if any, associated with $e^{Q_{0a}t}f$, we have to find the solutions of $Q'e^{Q_{0a}t}f=Q_{0a}e^{Q_{0a}t}f$ for $Q'\in\mathcal{Q}$.
The form of $Q'$ is
\begin{equation*} 
 Q'= 
 \begin{array}{|ccc|}
  -\lambda_{0} & \lambda_{0} & 0 \\
  \mu_{1} & -\mu_{1}-\lambda_{1} & \lambda_{1}  \\
  0 & \mu_{2} & -\mu{2}  
 \end{array}
 \end{equation*}
 where $\lambda_{i}\in[\underline{\lambda},\overline{\lambda}]$ and $\mu_{i+1}\in[\underline{\mu},\overline{\mu}]$, with $i\in\{0,1\}$.\\\\
 Therefore, $Q'e^{Q_{0a}t}f$ has the following form
 \begin{equation*} 
 Q'e^{Q_{0a}t}f= 
 \begin{array}{|c|}
  \lambda_{0}(e^{Q_{0a}t}f(1)-e^{Q_{0a}t}f(0))\\
  \mu_{1}(e^{Q_{0a}t}f(0)-e^{Q_{0a}t}f(1))+\lambda_{1}(e^{Q_{0a}t}f(2)-e^{Q_{0a}t}f(1))\\
  \mu_{2}(e^{Q_{0a}t}f(2)-e^{Q_{0a}t}f(1))  
 \end{array}
 \end{equation*}
We solve $Q'e^{Q_{0a}t}f=Q_{0a}e^{Q_{0a}t}f$ row-wise.
Taking $(Q'e^{Q_{0a}t}f)(0)=(Q_{0a}e^{Q_{0a}t}f)(0)$, we have that
\begin{equation*} 
\lambda_{0}(e^{Q_{0a}t}f(1)-e^{Q_{0a}t}f(0))=3(e^{Q_{0a}t}f(1)-e^{Q_{0a}t}f(0))
 \end{equation*}
which means that we actually need to find the first positive $t$, for which the function $e^{Q_{0a}t}f(1)-e^{Q_{0a}t}f(0)$ is not just zero, but also that the sign of the function at both sides of $t$ is different.
If the signs are different, then we know that $\lambda_{0}=1$, since before it was 3 in $Q_{0a}$.
Therefore, we need the first positive $t$, at which the function $e^{Q_{0a}t}f(i)-e^{Q_{0a}t}f(i+1)=0$ changes sign, for all $i\in\{0,\ldots,N-1\}$, with $N$ being the largest ordered element in the state space.\\\\
For $Q'e^{Q_{0a}t}f$, whose values are shown in \eqref{eQ0atf}, we do not have flipping times, because the functions
\begin{equation*}
\begin{split}
&e^{Q_{0a}t}f(0)-e^{Q_{0a}t}f(1)= 0.78867e^{-5.73205t}+0.21132e^{-2.26794t}\\
&e^{Q_{0a}t}f(1)-e^{Q_{0a}t}f(2)=-0.57735e^{-5.73205t}+0.57735e^{-2.26794t}
\end{split}
\end{equation*}
do not change sign for any $t>0$.\\\\
By taking $t\to\infty$, we have that
\begin{equation*} 
 f_{t}= 
 \begin{array}{|r|}
  0.307692\\
  0.307692\\
  0.307692
 \end{array}
 \end{equation*}
 %Matrix(3, 1, {(1, 1) = .3076923077, (2, 1) = .3076923077, (3, 1) = .3076923077})
By taking generator matrix 
\begin{equation*} 
 Q_{0b}= 
 \begin{array}{|rrr|}
  -3 & 3 & 0 \\
  2 & -3 & 1 \\
  0 & 5 & -5  
 \end{array}
 \end{equation*}
which is diagonalisable, we have that
\begin{equation*} 
 e^{Q_{0b}t}f = 
 \begin{array}{|r|}
  0.35714+0.14285e^{-7t}+0.50000e^{-4t}\\
  0.35714-0.19047e^{-7t}-0.16666e^{-4t}\\
  0.35714+0.47619e^{-7t}-0.83333e^{-4t}
 \end{array}
 \end{equation*}
%Matrix(3, 1, {(1, 1) = .3571428570+.1428571429*exp(-7.*t)+.5000000001*exp(-4.*t), (2, 1) = -.1904761904*exp(-7.*t)+.3571428572-.1666666668*exp(-4.*t), (3, 1) = -.8333333330*exp(-4.*t)+.4761904758*exp(-7.*t)+.3571428572})
which does not have a flipping time and by taking $t\to\infty$, we have that
\begin{equation*} 
 f_{t}= 
 \begin{array}{|r|}
  0.357142857\\
   0.357142857\\ 
   0.357142857
 \end{array}\\\\
 \end{equation*}
Similarly, by taking generator matrix 
\begin{equation*} 
 Q_{0c}= 
 \begin{array}{|rrr|}
  -3 & 3 & 0 \\
  2 & -5 & 3 \\
  0 & 2 & -2  
 \end{array}
 \end{equation*}
which is diagonalisable, we have that
\begin{equation*} 
 e^{Q_{0c}t}f = 
 \begin{array}{|r|}
  0.21052+0.20135e^{7.44948t}+0.58811e^{-2.55051t}\\
  0.21052-0.29864e^{7.44948t}+0.88117e^{-2.55051t}\\
  0.21052+0.10960e^{7.44948t}-0.32013e^{-2.55051t}
 \end{array}
 \end{equation*}
%Matrix(3, 1, {(1, 1) = .2105263157+.2013560733*exp(-7.449489743*t)+.5881176112*exp(-2.550510257*t), (2, 1) = -.2986439273*exp(-7.449489743*t)+.2105263161+0.8811761120e-1*exp(-2.550510257*t), (3, 1) = -.3201306792*exp(-2.550510257*t)+.1096043634*exp(-7.449489743*t)+.2105263158})
which does not have a flipping time and by taking $t\to\infty$, we have that
\begin{equation*} 
 f_{t}= 
 \begin{array}{|r|}
  0.210526315\\
  0.210526315\\  
  0.210526315
 \end{array}\\\\
 \end{equation*}
 Finally, by taking generator matrix 
\begin{equation*} 
 Q_{0d}= 
 \begin{array}{|rrr|}
  -3 & 3 & 0 \\
  2 & -5 & 3 \\
  0 & 5 & -5  
 \end{array}
 \end{equation*}
which is also diagonalisable, we have that
\begin{equation*} 
 e^{Q_{0d}t}f = 
 \begin{array}{|r|}   
  0.29411+0.07646e^{-9.37228t}+0.62941e^{-3.62771t}\\
  0.29411-0.16241e^{-9.37228t}-0.13169e^{-3.62771t}\\
  0.29411+0.18573e^{-9.37228t}-0.47985e^{-3.62771t}
 \end{array}
 \end{equation*}
%Matrix(3, 1, {(1, 1) = .2941176474+0.7646489910e-1*exp(-9.372281323*t)+.6294174535*exp(-3.627718677*t), (2, 1) = -.1624186166*exp(-9.372281323*t)+.2941176469-.1316990302*exp(-3.627718677*t), (3, 1) = -.4798543424*exp(-3.627718677*t)+.1857366954*exp(-9.372281323*t)+.2941176469})
which does not have a flipping time and by taking $t\to\infty$, we have that
\begin{equation*} 
 f_{t}= 
 \begin{array}{|r|}
  0.294117647\\
  0.294117647\\
  0.294117647
 \end{array}\\\\
 \end{equation*} 
The lower probability $\underline{E}(X=0)$ is $0.210526315$ and is associated with generator matrix $Q_{0c}$.
The result is reasonable, as in the discrete time case, our generator matrix is consists of the largest birth rate and the smallest death rate at each row.\\\\

***Upper probability ($\overline{E}(X=0)$)***\\\\
Similarly, for the upper probability, we have that $\overline{E}(X=0)= 0.8064516129$, which obtained by a single generator matrix, which consists of the smallest birth rate and the largest death rate at each row, that is
\begin{equation*} 
 \overline{Q}_{0}= 
 \begin{array}{|rrr|}
  -1 & 1 & 0 \\
  5 & -6 & 1 \\
  0 & 5 & -5  
 \end{array}
\end{equation*}

--- \textbf{State 2} ($f\coloneqq[0,0,1]^T$) ---\\

The procedure is the same as previously and we expectedly obtain symmetrical results\\

***Lower probability ($\underline{E}(X=2)$)***\\

There are no flipping times and the generator matrix is 
\begin{equation*} 
 \underline{Q}_{0}= 
 \begin{array}{|rrr|}
  -1 & 1 & 0 \\
  5 & -6 & 1 \\
  0 & 5 & -5  
 \end{array}
\end{equation*}
%Matrix([[-1, 1, 0], [5, -6, 1], [0, 5, -5]]), $\lambda_{i}=\underline{\lambda}=1$ and $\mu_{i}=\overline{\mu}=5$
from which we get $\underline{E}(X=2)=0.03225806452$.\\\\

***Upper probability ($\overline{E}(X=2)$)***\\

There are no flipping times and the generator matrix is 
\begin{equation*} 
 \overline{Q}_{0}= 
 \begin{array}{|rrr|}
  -3 & 3 & 0 \\
  2 & -5 & 3 \\
  0 & 2 & -2  
 \end{array}
\end{equation*}
% Matrix([[-3, 3, 0], [2, -5, 3], [0, 2, -2]]), i.e $\lambda_{i}=\overline{\lambda}=3$ and $\mu_{i}=\underline{\mu}=2$
from which we get $\overline{E}(X=2)= 0.4736841978$.\\\\\\

--- \textbf{State 1} ($f\coloneqq[0,1,0]^T$) ---\\

***Lower probability ($\overline{E}(X=1)$)***\\

--- First Iteration ---\\
In this case, we can directly find $\underline{Q}_{0}$, which is
\begin{equation*} 
 \underline{Q}_{0}= 
 \begin{array}{|rrr|}
  -1 & 1 & 0 \\
  5 & -8 & 3 \\
  0 & 2 & -2 \\
 \end{array}
 \end{equation*}
% Q0a := Matrix([[-1.0, 1.0, 0., 0.], [5.0, -8.0, 3.0, 0.], [0., 2.0, -5.0, 3.0], [0., 0., 2.0, -2.0]])
which is diagonalisable and yields the following
\begin{equation*} 
 e^{\underline{Q}_{0}t}f= 
 \begin{array}{|r|}
  0.13333-0.03252e^{-1.59487t}-0.10080e^{-9.40512t}\\
  0.13333+0.01934e^{-1.59487t}+0.84731e^{-9.40512t}\\
  0.13333+0.09551e^{-1.59487t}-0.22884e^{-9.40512t}
 \end{array}
 \end{equation*}
% eQ0tf:=Matrix(3, 1, {(1, 1) = -0.3252349872e-1*exp(-1.594875162*t)-.1008098346*exp(-9.405124838*t)+.1333333333, (2, 1) = .1333333329+0.1934742180e-1*exp(-1.594875162*t)+.8473192450*exp(-9.405124838*t), (3, 1) = -.2288467146*exp(-9.405124838*t)+.1333333334+0.9551338120e-1*exp(-1.594875162*t)})
for the values above, the flipping time is $t_{1}:= 0.339073069$, where $\lambda_{1}$ from $3$ becomes $1$ and $\mu_{2}$ from $2$ becomes $5$.\\\\
The updated gamble at time $t_{1}$ is
\begin{equation*}  
 f_{t_1}= 
 \begin{array}{|c|}
  0.1102405\\
  0.1795189\\
  0.1795189
 \end{array}
 %Matrix(3, 1, {(1, 1) = .1102405039, (2, 1) = .1795189918, (3, 1) = .1795189923})
 \end{equation*}\\\\

--- Second Iteration ---\\
Now the generator matrix of the process is
\begin{equation*} 
 \underline{Q}_{1}= 
 \begin{array}{|rrr|}
  -1 & 1 & 0 \\
  5 & -6 & 1 \\
  0 & 5 & -5 
 \end{array}
 \end{equation*}
 %Matrix([[-1, 1, 0], [5, -6, 1], [0, 5, -5]]),
according to the changes in $\lambda_{1}$ and $\mu_{2}$.
Generator matrix $\underline{Q}_1$ is diagonalisable and yields 
\begin{equation*} 
 e^{\underline{Q}_{1}t}f_{t_1}= 
 \begin{array}{|r|}
 0.123649-0.004205e^{-8.23606t}-0.009202e^{-3.76393t}\\
 0.123649+0.030433e^{-8.23606t}+0.025436e^{-3.76393t}\\
 0.123649-0.047022e^{-8.23606t}+0.102892e^{-3.76393t}
 \end{array}
 \end{equation*}
% Matrix(3, 1, {(1, 1) = .1236492435-0.420579868e-2*exp(-8.236067978*t)-0.920294094e-2*exp(-3.763932022*t), (2, 1) = 0.3043344521e-1*exp(-8.236067978*t)+.1236492434+0.2543630324e-1*exp(-3.763932022*t), (3, 1) = .1028920075*exp(-3.763932022*t)-0.470222587e-1*exp(-8.236067978*t)+.1236492436})
For the matrix above there is no flipping time and by taking $t\to\infty$, we have that
$\underline{E}(X=1) = 0.123649243$.\\\\\\


***Upper probability ($\overline{E}(X=1)$)***\\

--- First Iteration ---\\
In this case, we can directly find $\underline{Q}_{0}$, which is
\begin{equation*} 
 \overline{Q}_{0}= 
 \begin{array}{|rrr|}
  -3 & 3 & 0 \\
  2 & -3 & 1 \\
  0 & 5 & -5 \\
 \end{array}
 \end{equation*}
which is diagonalisable and yields the following
\begin{equation*} 
 e^{\overline{Q}_{0}t}f= 
 \begin{array}{|r|}
  0.535714-0.24999e^{-4t}-0.28571e^{-7t}\\
  0.535714+0.08333e^{-4t}+0.38095e^{-7t}\\
  0.535714+0.416661e^{-4t}-0.95238e^{-7t}
 \end{array}
 \end{equation*}
% eQ0tf:=Matrix(3, 1, {(1, 1) = -.2499999999*exp(-4.*t)-.2857142859*exp(-7.*t)+.5357142858, (2, 1) = .5357142852+0.8333333330e-1*exp(-4.*t)+.3809523811*exp(-7.*t), (3, 1) = -.9523809525*exp(-7.*t)+.5357142860+.4166666670*exp(-4.*t)})
for the values above, the flipping time is $t_{1}:= 0.4620981202$, where $\lambda_{1}$ from $1$ becomes $3$ and $\mu_{2}$ from $5$ becomes $2$.\\\\
The updated gamble at time $t_{1}$ is
\begin{equation*}  
 f_{t_1}= 
 \begin{array}{|c|}
  0.485092\\
  0.563837\\
  0.563837
 \end{array}
 %Matrix(3, 1, {(1, 1) = .4850924579, (2, 1) = .5638375230, (3, 1) = .5638375237})
 \end{equation*}\\\\

--- Second Iteration ---\\
Now the generator matrix of the process is
\begin{equation*} 
 \overline{Q}_{1}= 
 \begin{array}{|rrr|}
  -3 & 3 & 0 \\
  2 & -5 & 3 \\
  0 & 2 & -2 
 \end{array}
 \end{equation*}
 %Matrix([[-3, 3, 0], [2, -5, 3], [0, 2, -2]])
according to the changes in $\lambda_{1}$ and $\mu_{2}$.
Generator matrix $\overline{Q}_1$ is diagonalisable and yields 
\begin{equation*} 
 e^{\overline{Q}_{1}t}f_{t_1}= 
 \begin{array}{|r|}
 0.547259-0.015856e^{-7.44948t}-0.046311e^{-2.55051t}\\
 0.547259+0.023516e^{-7.44948t}-0.006938e^{-2.55051t}\\
 0.547259-0.008630e^{-7.44948t}+0.025208e^{-2.55051t}
 \end{array}
 \end{equation*}
% Matrix(3, 1, {(1, 1) = .5472596149-0.158557969e-1*exp(-7.449489743*t)-0.4631136002e-1*exp(-2.550510257*t), (2, 1) = 0.235167354e-1*exp(-7.449489743*t)+.5472596152-0.693882739e-2*exp(-2.550510257*t), (3, 1) = 0.2520871175e-1*exp(-2.550510257*t)-0.86308027e-2*exp(-7.449489743*t)+.5472596150})
For the matrix above there is no flipping time and by taking $t\to\infty$, we have that
$\overline{E}(X=1) = 0.547259615$.\\\\\\\\
\textbf{Lower and Upper probabilities for states 1 and 2 when the state space is $\stateset:=\{0,1,2,3\}$}(The lower and upper probabilities of states 0 and 3 follow are computed similarly, due to monotonicity, to the respective ones of 0 and 2 for the state space is $\{0,1,2\}$)\\


% State 0, i.e. $f=[1,0,0,0]^{T}$

% -----------

% Lower probability = 0.2105263158 -- No flipping time, Matrix([[-3, 3, 0], [2, -5, 3], [0, 2, -2]]), i.e $\lambda_{i}=\overline{\lambda}=3$ and $\mu_{i}=\underline{\mu}=2$ for all $i\in\stateset$\\

% Upper probability = 0.8064516129 -- No flipping time-Matrix([[-1, 1, 0], [5, -6, 1], [0, 5, -5]]), i.e $\lambda_{i}=\underline{\lambda}=1$ and $\mu_{i}=\overline{\mu}=5$ for all $i\in\stateset$\\

--- \textbf{State 1} ($f\coloneqq[0,1,0,0]^T$) ---\\

***Lower probability ($\underline{E}(X=1)$)***\\

--- First Iteration ---\\
Again we encounter the problem that we do not know directly which generator matrix $Q$ minimises $Qf$.
We know that the form of the initial generator matrix, $\underline{Q}_{0}$, should be
\begin{equation*} 
 \underline{Q}_{0}= 
 \begin{array}{|rrrr|}
  -1 & 1 & 0 & 0 \\
  5 & -8 & 3 & 0 \\
  0 & 2 & ? & ? \\
  0 & 0 & ? & ? 
 \end{array}
 \end{equation*}
 where for the unknown values (?) we test the possible combinations of extreme points $\{\underline{\lambda},\overline{\lambda}\}\times\{\underline{\mu},\overline{\mu}\}$.\\\\\\
We start with the generator matrix 
\begin{equation*} 
 Q_{0a}= 
 \begin{array}{|rrrr|}
  -1 & 1 & 0 & 0 \\
  5 & -8 & 3 & 0 \\
  0 & 2 & -5 & 3 \\
  0 & 0 & 5 & -5 
 \end{array}
 \end{equation*}
% Q0a := Matrix([[-1.0, 1.0, 0., 0.], [5.0, -8.0, 3.0, 0.], [0., 2.0, -5.0, 3.0], [0., 0., 5.0, -5.0]])
% Eigenvectors and their inverses
% [[-0.5         0.31514304 -0.07636804  0.10298864]
%  [-0.5        -0.02596395  0.73221067 -0.65188641]
%  [-0.5        -0.58510792 -0.50435412 -0.31730256]
%  [-0.5        -0.746766    0.45128968  0.68099672]]
% [[-1.19047619 -0.23809524 -0.35714286 -0.21428571]
%  [ 1.04167794 -0.01716431 -0.58020701 -0.44430662]
%  [-0.33786726  0.64788884 -0.66940858  0.35938699]
%  [ 0.49211336 -0.62298522 -0.45485258  0.58572444]]
which is diagonalisable and yields the following
\begin{equation*} 
 e^{Q_{0a}t}f= 
 \begin{array}{|r|}
  0.119047-0.04947e^{-10.58791t}-0.06416e^{-7.32969t}-0.00540e^{-1.08238t}\\
  0.119047+0.47439e^{-10.58791t}+0.40611e^{-7.32969t}+ 0.00044e^{-1.08238t}  \\
  0.119047-0.32676e^{-10.58791t}+ 0.19767e^{-7.32969t}+ 0.01004e^{-1.08238t}  \\
  0.119047+0.29238e^{-10.58791t}- 0.42425e^{-7.32969t}+ 0.01281e^{-1.08238t}  
 \end{array}
 \end{equation*}
% eQ0tf:=[[ -0.0494780026757232$e^{-10.5879198806914t}$ - 0.0641604019326391$e$(-7.32969229967929*t) - 0.00540921443925667$e$(-1.08238781962931*t) + 0.119047619047619$e$(-2.22044604925031e-16*t)]
% [ 0.474391125511469$e$(-10.5879198806914*t) + 0.406115602057355$e$(-7.32969229967929*t) + 0.000445653383557735$e$(-1.08238781962931*t) + 0.119047619047619$e$(-2.22044604925031e-16*t)]
% [ -0.326765403852027$e$(-10.5879198806914*t) + 0.197674808314207$e$(-7.32969229967929*t) + 0.0100429764902019$e$(-1.08238781962931*t) + 0.119047619047619$e$(-2.22044604925031e-16*t)]
% [ 0.292385548494654$e$(-10.5879198806914*t) - 0.424250894294965$e$(-7.32969229967929*t) + 0.012817726752692$e$(-1.08238781962931*t) + 0.119047619047619$e$(-2.22044604925031e-16*t)]]
for the values above, the flipping time is $t_{1}:= 0.5682803520$, where $\lambda_{1}$ from $3$ becomes $1$ and $\mu_{2}$ from $2$ becomes $5$.\\\\
The updated gamble at time $t_{1}:= 0.5682803520$ is
\begin{equation*}  
 f_{t_1}= 
 \begin{array}{|c|}
  0.115006\\
  0.126749\\
  0.126749\\
  0.120102  
 \end{array}
 \end{equation*}\\\\\\
% t1:= 0.5682803520, f(1) and f(2), hence l1 and m2 change
%updated gamble = Matrix(4, 1, {(1, 1) = .1150068066+0.1382563053e-14*I, (2, 1) = .1267495599-0.4094226054e-13*I, (3, 1) = .1267491807+0.7771686030e-13*I, (4, 1) = .1201029295-0.9171760687e-13*I})
We take now generator matrix 
\begin{equation*} 
 Q_{0b}= 
 \begin{array}{|rrrr|}
  -1 & 1 & 0 & 0 \\
  5 & -8 & 3 & 0 \\
  0 & 2 & -5 & 3 \\
  0 & 0 & 2 & -2 
 \end{array}
 \end{equation*}
 which is also diagonalisable and yields
 % Q0b := Matrix([[-1,1,0,0],[5,-8,3,0],[0,2,-5,3],[0,0,2,-2]])
\begin{equation*} 
 e^{Q_{0b}t}f= 
 \begin{array}{|r|}
  0.10256-0.07790e^{-9.97710t}- 0.04501e^{-5.28298t}+ 0.02035e^{-0.73991t}\\
  0.10256+0.69936e^{-9.97710t}+ 0.19277e^{-5.28298t}+ 0.00529e^{-0.73991t}\\
  0.10256-0.33106e^{-9.97710t}+ 0.24960e^{-5.28298t}- 0.02110e^{-0.73991t}\\
  0.10256+0.08300e^{-9.97710t}- 0.15206e^{-5.28298t}- 0.03350e^{-0.73991t}  
 \end{array}
 \end{equation*}
 % eQ0tf:=[[ -0.0779054769043385$e$(-9.97710623307476*t) - 0.0450100230586763$e$(-5.28298012668265*t) + 0.0203513973989123$e$(-0.739913640242596*t) + 0.102564102564102$e$(2.57459787482833e-17*t)]
 %[ 0.699365742308599$e$(-9.97710623307476*t) + 0.192777034261839$e$(-5.28298012668265*t) + 0.00529312086545942$e$(-0.739913640242596*t) + 0.102564102564102$e$(2.57459787482833e-17*t)]
 %[ -0.331064327931864$e$(-9.97710623307476*t) + 0.249609716167326$e$(-5.28298012668265*t) - 0.0211094907995641$e$(-0.739913640242596*t) + 0.102564102564102$e$(2.57459787482833e-17*t)]
 %[ 0.0830036151603954$e$(-9.97710623307476*t) - 0.152062885875309$e$(-5.28298012668265*t) - 0.0335048318491888$e$(-0.739913640242596*t) + 0.102564102564102$e$(2.57459787482833e-17*t)]]
for the values above, the flipping time is $t_{1}:= 0.6403990997$, where $\lambda_{0}$ from $1$ becomes $3$ and $\mu_{1}$ from $5$ becomes $2$.\\\\
The updated gamble at time $t_{1}:= 0.6403990997$ is
\begin{equation} \label{eq:ind_s1_4_f1}
 f_{t_1}= 
 \begin{array}{|c|}
  0.113576\\
  0.113576\\
  0.097336\\
  0.076682  
 \end{array}
 \end{equation}\\\\\\
 % t1:= 0.6403990997, f(0) and f(1), hence l0 and m1 change
%updated gamble = Matrix(4, 1, {(1, 1) = .1135766323-0.2733214480e-12*I, (2, 1) = .1135766323+0.2173956598e-11*I, (3, 1) = 0.9733649580e-1-0.4689821958e-12*I, (4, 1) = 0.7668242781e-1-0.4616713694e-13*I})
We take now generator matrix 
\begin{equation*} 
 Q_{0c}= 
 \begin{array}{|rrrr|}
  -1 & 1 & 0 & 0 \\
  5 & -8 & 3 & 0 \\
  0 & 2 & -3 & 1 \\
  0 & 0 & 5 & -5 
 \end{array}
 \end{equation*}
 which is also diagonalisable and yields
 % Q0c = Matrix([[-1,1,0,0],[5,-8,3,0],[0,2,-3,1],[0,0,5,-5]])
\begin{equation*} 
 e^{Q_{0c}t}f= 
 \begin{array}{|r|}
  0.12820-0.08861e^{-9.65331t}- 0.01960e^{-6t}- 0.01998e^{-1.34668t}\\
  0.12820+0.76682e^{-9.65331t}+ 0.09803e^{-6t}+ 0.00692e^{-1.34668t}\\
  0.12820-0.27490e^{-9.65331t}+ 0.09803e^{-6t}+ 0.04866e^{-1.34668t}\\
  0.12820+0.29538e^{-9.65331t}- 0.49019e^{-6t}+ 0.06660e^{-1.34668t}  
 \end{array}
 \end{equation*}
%  eQ0tf:=[[ -0.0886167816488844$e$(-9.65331193145905*t) - 0.0196078431372549$e$(-6.0*t) - 0.0199805034189889$e$(-1.34668806854096*t) + 0.128205128205128$e$(2.22044604925031e-16*t)]
% [ 0.766828653969792$e$(-9.65331193145905*t) + 0.0980392156862746$e$(-6.0*t) + 0.00692700213880534$e$(-1.34668806854096*t) + 0.128205128205128$e$(2.22044604925031e-16*t)]
% [ -0.274907684916169$e$(-9.65331193145905*t) + 0.0980392156862744$e$(-6.0*t) + 0.0486633410247667$e$(-1.34668806854096*t) + 0.128205128205128$e$(2.22044604925031e-16*t)]
% [ 0.295389272162948$e$(-9.65331193145905*t) - 0.490196078431372$e$(-6.0*t) + 0.0666016780632961$e$(-1.34668806854096*t) + 0.128205128205128$e$(2.22044604925031e-16*t)]]
for the values above, the flipping time is $t_{1}:= 0.38731403185$, where $\lambda_{1}$ from $3$ becomes $1$ and $\mu_{2}$ from $2$ becomes $5$.\\\\
The updated gamble at time $t_{1}:= 0.38731403185$ is
\begin{equation*} 
 f_{t_1}= 
 \begin{array}{|c|}
  0.112318\\
  0.16015\\
  0.16015\\
  0.126776  
 \end{array}
 \end{equation*}\\\\\\
 % t1:= 0.38731403185, f(1) and f(2), hence l1 and m2 change updated gamble = Matrix(4, 1, {(1, 1) = .1123183253, (2, 1) = .1601501863, (3, 1) = .1601501872, (4, 1) = .1267763536})
 The last generator matrix we have to consider is
\begin{equation*} 
 Q_{0d}= 
 \begin{array}{|rrrr|}
  -1 & 1 & 0 & 0 \\
  5 & -8 & 3 & 0 \\
  0 & 2 & -3 & 1 \\
  0 & 0 & 2 & -2 
 \end{array}
 \end{equation*}
 which is also diagonalisable and yields
 % Q0d = np.array([[-1,1,0,0],[5,-8,3,0],[0,2,-3,1],[0,0,2,-2]])
\begin{equation*} 
 e^{Q_{0d}t}f= 
 \begin{array}{|r|}
 0.12121-0.09547e^{-9.54138t}- 0.02573e^{-3.45861t}\\
  0.12121+0.81551e^{-9.54138t}+ 0.06326e^{-3.45861t}\\
  0.12121-0.25987e^{-9.54138t}+ 0.13866e^{-3.45861t}\\
  0.12121+0.06892e^{-9.54138t}- 0.19013e^{-3.45861t}
 \end{array}
 \end{equation*}
%  eQ0tf:=[[ -0.0954785730647727$e$(-9.54138126514911*t) - 0.0257335481473485$e$(-3.45861873485089*t) - 2.157944886736e-17$e$(-1.0*t) + 0.121212121212121$e$(5.1635269239109e-16*t)]
% [ 0.815518895198621$e$(-9.54138126514911*t) + 0.0632689835892584$e$(-3.45861873485089*t) + 8.92606466687637e-33$e$(-1.0*t) + 0.121212121212121$e$(5.1635269239109e-16*t)]
% [ -0.25987756037013$e$(-9.54138126514911*t) + 0.138665439158009$e$(-3.45861873485089*t) + 3.59657481122666e-17$e$(-1.0*t) + 0.121212121212121$e$(5.1635269239109e-16*t)]
% [ 0.0689204142405846$e$(-9.54138126514911*t) - 0.190132535452706$e$(-3.45861873485089*t) + 7.19314962245332e-17$e$(-1.0*t) + 0.121212121212121$e$(5.1635269239109e-16*t)]]
for the values above, the flipping time is $t_{1}:= 0.4369206243$, where $\lambda_{1}$ from $3$ becomes $1$ and $\mu_{2}$ from $2$ becomes $5$.\\\\
The updated gamble at time $t_{1}:= 0.4369206243$ is
\begin{equation*} 
 f_{t_1}= 
 \begin{array}{|c|}
  0.114056\\
  0.147789\\
  0.147789\\
  0.080324  
 \end{array}
 \end{equation*}\\\\\\
% t1:= 0.4369206243, f(1) and f(2), hence l1 and m2 change updated gamble = Matrix(4, 1, {(1, 1) = .1140567187, (2, 1) = .1477893305, (3, 1) = .1477893302, (4, 1) = 0.8032410728e-1})

--- Second Iteration ---\\
The minimum, row-wise, vector is the $f_{t_1}$ from \eqref{eq:ind_s1_4_f1} associated with generator matrix $\underline{Q}_{0}=Q_{0b}$.
Now the generator matrix of the process is
\begin{equation*} 
 \underline{Q}_{1}= 
 \begin{array}{|rrrr|}
  -3 & 3 & 0 & 0 \\
  2 & -5 & 3 & 0 \\
  0 & 2 & -5 & 3 \\
  0 & 0 & 2 & -2 
 \end{array}
 \end{equation*}
according to the changes in $\lambda_{0}$ and $\mu_{1}$, at flipping time point $t_{1}\coloneqq 0.6403990997$.
Generator matrix $\underline{Q}_1$ is diagonalisable and yields 
\begin{equation*} 
 e^{\underline{Q}_{1}t}f_{t_1}= 
 \begin{array}{|r|}
 0.093754+0.002527e^{-8.46410t}-0.009294e^{-5t}+0.028864e^{-1.53589t}\\
 0.093754-0.000460e^{-8.46410t}+0.006196e^{-5t}+0.014086e^{-1.53589t}\\
 0.093754+0.000363e^{-8.46410t}+0.006196e^{-5t}-0.002976e^{-1.53589t}\\
 0.093754-0.000112e^{-8.46410t}-0.004130e^{-5t}-0.012828e^{-1.53589t}  
 \end{array}
 \end{equation*}
% (from Q0b) Q1 := Matrix([[-3,3,0,0],[2,-5,3,0],[0,2,-5,3],[0,0,2,-2]])
% Matrix(4, 1, {(1, 1) = 0.9375407882e-1+0.25274635e-3*exp(-8.464101615*t)-0.929433068e-2*exp(-5.*t)+0.2886413789e-1*exp(-1.535898385*t), (2, 1) = -0.46034388e-3*exp(-8.464101615*t)+0.9375407874e-1+0.619622048e-2*exp(-5.*t)+0.1408667698e-1*exp(-1.535898385*t), (3, 1) = 0.619622043e-2*exp(-5.*t)+0.36306168e-3*exp(-8.464101615*t)+0.9375407880e-1-0.2976865208e-2*exp(-1.535898385*t), (4, 1) = -0.1282850569e-1*exp(-1.535898385*t)-0.413081362e-2*exp(-5.*t)-0.11233166e-3*exp(-8.464101615*t)+0.9375407877e-1})
% no flipping time t2
%lower prob = 0.0937540788
For the matrix above there is no flipping time and by taking $t\to\infty$, we have that
$\underline{E}(X=1) = 0.0937540788$.\\\\\\


***Upper probability ($\overline{E}(X=1)$)***\\

--- First Iteration ---

Again we encounter the problem that we do not know directly which generator matrix $Q$ minimises $Qf$.
We know that the form of the initial generator matrix should be
\begin{equation*} 
 \overline{Q}_{0}= 
 \begin{array}{|rrrr|}
  -3 & 3 & 0 & 0 \\
  2 & -3 & 1 & 0 \\
  0 & 5 & ? & ? \\
  0 & 0 & ? & ? 
 \end{array}
 \end{equation*}\\\\
We start with the generator matrix 
\begin{equation*} 
 Q_{0a}= 
 \begin{array}{|rrrr|}
  -3 & 3 & 0 & 0 \\
  2 & -3 & 1 & 0 \\
  0 & 5 & -8 & 3 \\
  0 & 0 & 5 & -5 
 \end{array}
 \end{equation*}
% Q0a := Matrix([[-3,3,0,0],[2,-3,1,0],[0,5,-8,3],[0,0,5,-5])
which is diagonalisable and yields the following
\begin{equation*} 
 e^{Q_{0a}t}f= 
 \begin{array}{|r|}
  0.50335-0.02255e^{-11.12559t}- 0.53512e^{-5.38945t}+ 0.05432e^{-2.48495t}\\
  0.50335+0.06109e^{-11.12559t}+ 0.42622e^{-5.38945t}+ 0.00932e^{-2.48495t}\\
  0.50335-0.45132e^{-11.12559t}+ 0.05181e^{-5.38945t}- 0.10385e^{-2.48495t}\\
  0.50335+0.36838e^{-11.12559t}- 0.66527e^{-5.38945t}- 0.20646e^{-2.48495t}  
 \end{array}
 \end{equation*}
% eQ0tf:= [[ -0.0225565178214933$e$(-11.1255911947135*t) - 0.535128593563363$e$(-5.38945499254273*t) + 0.0543294066868703$e$(-2.48495381274377*t) + 0.503355704697986$e$(1.04980154561764e-15*t)]
%[ 0.0610950141979078$e$(-11.1255911947135*t) + 0.426221896514117$e$(-5.38945499254273*t) + 0.00932738458998848$e$(-2.48495381274377*t) + 0.503355704697987$e$(1.04980154561764e-15*t)]
% [ -0.45132007376443$e$(-11.1255911947135*t) + 0.0518191485700378$e$(-5.38945499254273*t) - 0.103854779503595$e$(-2.48495381274377*t) + 0.503355704697986$e$(1.04980154561764e-15*t)]
% [ 0.368388992522002$e$(-11.1255911947135*t) - 0.665277754326795$e$(-5.38945499254273*t) - 0.206466942893193$e$(-2.48495381274377*t) + 0.503355704697986$e$(1.04980154561764e-15*t)]]
for the values above, the flipping time is $t_{1}:= 1.054169799$, where $\lambda_{2}$ from $3$ becomes $1$ and $\mu_{3}$ from $5$ becomes $2$.\\\\
The updated gamble at time $t_{1}:= 1.054169799$ is
\begin{equation*}  
 f_{t_1}= 
 \begin{array}{|c|}
  0.505488\\
  0.505488\\
  0.495964\\
  0.486052  
 \end{array}
 \end{equation*}\\\\\\
% t1:= 1.054169799, f(0) and f(1), hence l0 and m1 change
%updated gamble = Matrix(4, 1, {(1, 1) = .5054884511-0.4470076635e-13*I, (2, 1) = .5054884512+0.4812107722e-13*I, (3, 1) = .4959644775-0.8410133317e-13*I, (4, 1) = .4860529374-0.1250249744e-13*I})
We consider now the generator matrix 
\begin{equation*} 
 Q_{0b}= 
 \begin{array}{|rrrr|}
  -3 & 3 & 0 & 0 \\
  2 & -3 & 1 & 0 \\
  0 & 5 & -8 & 3 \\
  0 & 0 & 2 & -2 
 \end{array}
 \end{equation*}
% Q0b := Matrix([[-3.0, 3.0, 0., 0.], [2.0, -3.0, 1.0, 0.], [0., 5.0, -8.0, 3.0], [0., 0., 2.0, -2.0]])
which is diagonalisable and yields the following
\begin{equation*} 
 e^{Q_{0b}t}f= 
 \begin{array}{|r|}
  0.46153-0.05349e^{-9.65331t}- 0.52941e^{-5t}+ 0.12136e^{-1.34668t}\\
  0.46153+0.11863e^{-9.65331t}+ 0.35294e^{-5t}+ 0.06688e^{-1.34668t}\\
  0.46153-0.68232e^{-9.65331t}+ 0.35294e^{-5t}- 0.13214e^{-1.34668t}\\
  0.46153+0.17830e^{-9.65331t}- 0.23529e^{-5t}- 0.40455e^{-1.34668t}  
 \end{array}
 \end{equation*}
% eQ0tf:= [[ -0.0534928933043711$e$(-9.65331193145904*t) - 0.529411764705882$e$(-5.0*t) + 0.121366196471792$e$(-1.34668806854096*t) + 0.461538461538462$e$(1.37435128030003e-16*t)]
% [ 0.118634968423412$e$(-9.65331193145904*t) + 0.352941176470588$e$(-5.0*t) + 0.0668853935675382$e$(-1.34668806854096*t) + 0.461538461538461$e$(1.37435128030003e-16*t)]
% [ -0.682329664291011$e$(-9.65331193145904*t) + 0.352941176470589$e$(-5.0*t) - 0.132149973718039$e$(-1.34668806854096*t) + 0.461538461538461$e$(1.37435128030003e-16*t)]
% [ 0.178309644347903$e$(-9.65331193145904*t) - 0.235294117647059$e$(-5.0*t) - 0.404553988239305$e$(-1.34668806854096*t) + 0.461538461538462$e$(1.37435128030003e-16*t)]]
for the values above, the flipping time is $t_{1}:= 0.7637751743$, where $\lambda_{0}$ from $3$ becomes $1$ and $\mu_{1}$ from $2$ becomes $5$.\\\\
The updated gamble at time $t_{1}:= 0.7637751743$ is
\begin{equation*} 
 f_{t_1}= 
 \begin{array}{|c|}
  0.493273\\
  0.493273\\
  0.421611\\
  0.311849  
 \end{array}
 \end{equation*}\\\\\\
% t1:= 0.7637751743, f(0) and f(1), hence l0 and m1 change
%updated gamble = Matrix(4, 1, {(1, 1) = .4932736900, (2, 1) = .4932736895, (3, 1) = .4216117632, (4, 1) = .3118494387})
We consider now the generator matrix 
\begin{equation*} 
 Q_{0c}= 
 \begin{array}{|rrrr|}
  -3 & 3 & 0 & 0 \\
  2 & -3 & 1 & 0 \\
  0 & 5 & -6 & 1 \\
  0 & 0 & 5 & -5 
 \end{array}
 \end{equation*}
% Q0c = Matrix([[-3.0, 3.0, 0., 0.], [2.0, -3.0, 1.0, 0.], [0., 5.0, -6.0, 1.0], [0., 0., 5.0, -5.0]])
which is diagonalisable and yields the following
\begin{equation*} 
 e^{Q_{0c}t}f= 
 \begin{array}{|r|}
  0.52447-0.08031e^{-8.53662t}- 0.42431e^{-5.30677t}- 0.01984e^{-3.15659t}\\
  0.52447+0.14821e^{-8.53662t}+ 0.32626e^{-5.30677t}+ 0.00103e^{-3.15659t}\\
  0.52447-0.66001e^{-8.53662t}+ 0.09600e^{-5.30677t} + 0.03952e^{-3.15659t}\\
  0.52447+0.93310e^{-8.53662t}- 1.56479e^{-5.30677t}+ 0.10721e^{-3.15659t}  
 \end{array}
 \end{equation*}
% [[ -0.0803118712920433$e$(-8.53662962120579*t) - 0.424319164164894$e$(-5.30677486612787*t) - 0.0198444890185866$e$(-3.15659551266633*t) + 0.524475524475524$e$(8.88178419700125e-16*t)]
% [ 0.148219028509999$e$(-8.53662962120579*t) + 0.326269594370655$e$(-5.30677486612787*t) + 0.00103585264382229$e$(-3.15659551266633*t) + 0.524475524475524$e$(8.88178419700125e-16*t)]
% [ -0.660010121090719$e$(-8.53662962120579*t) + 0.0960078284538271$e$(-5.30677486612787*t) + 0.0395267681613671$e$(-3.15659551266633*t) + 0.524475524475524$e$(8.88178419700125e-16*t)]
% [ 0.933106080904353$e$(-8.53662962120579*t) - 1.56479293212059$e$(-5.30677486612787*t) + 0.107211326740718$e$(-3.15659551266633*t) + 0.524475524475524$e$(8.88178419700125e-16*t)]]
for the values above, the flipping time is $t_{1}:= 0.910933094$, where $\lambda_{1}$ from $1$ becomes $3$ and $\mu_{2}$ from $5$ becomes $2$.\\\\
The updated gamble at time $t_{1}:= 0.910933094$ is
\begin{equation} \label{eq:ind_s1_4_f1_u}
 f_{t_1}= 
 \begin{array}{|c|}
  0.51994\\
  0.52719\\
  0.52719\\
  0.51846  
 \end{array}
 \end{equation}\\\\\\
% t1:= .910933094, f(1) and f(2), hence l1 and m2 change
%updated gamble:=Matrix(4, 1, {(1, 1) = .5199478686-0.1009007918e-14*I, (2, 1) = .5271911782-0.2412942452e-15*I, (3, 1) = .5271911773+0.6837253339e-14*I, (4, 1) = .5184668566-0.1133711177e-13*I})
We consider now the generator matrix 
\begin{equation*} 
 Q_{0d}= 
 \begin{array}{|rrrr|}
  -3 & 3 & 0 & 0 \\
  2 & -3 & 1 & 0 \\
  0 & 5 & -6 & 1 \\
  0 & 0 & 2 & -2 
 \end{array}
 \end{equation*}
% Q0d = Matrix([[-3.0, 3.0, 0., 0.], [2.0, -3.0, 1.0, 0.], [0., 5.0, -6.0, 1.0], [0., 0., 2.0, -2.0]])
which is diagonalisable and yields the following
\begin{equation*} 
 e^{Q_{0d}t}f= 
 \begin{array}{|r|}
  0.50847-0.16212e^{-7.77016t}- 0.40086e^{-4.56735t}+ 0.05451e^{-1.66248t}\\
  0.50847+0.25778e^{-7.77016t}+ 0.20943e^{-4.56735t}+ 0.02430e^{-1.66248t}\\
  0.50847-0.90542e^{-7.77016t}+ 0.47347e^{-4.56735t} - 0.07652e^{-1.66248t}\\
  0.50847+0.31383e^{-7.77016t}- 0.36884e^{-4.56735t}- 0.45345e^{-1.66248t}  
 \end{array}
 \end{equation*}
% eQ0tf:= [[ -0.162123267975571$e$(-7.77016191034857*t) - 0.400868827030092$e$(-4.56735537309151*t) + 0.0545175187344769$e$(-1.66248271655993*t) + 0.508474576271187$e$(-1.55504366571219e-16*t)]
% [ 0.257784745892768$e$(-7.77016191034857*t) + 0.209434636650168$e$(-4.56735537309151*t) + 0.0243060411858769$e$(-1.66248271655993*t) + 0.508474576271187$e$(-1.55504366571219e-16*t)]
% [ -0.905428439975424$e$(-7.77016191034857*t) + 0.473479150995075$e$(-4.56735537309151*t) - 0.0765252872908373$e$(-1.66248271655993*t) + 0.508474576271186$e$(-1.55504366571219e-16*t)]
% [ 0.313831207526975$e$(-7.77016191034857*t) - 0.368845821624554$e$(-4.56735537309151*t) - 0.453459962173608$e$(-1.66248271655993*t) + 0.508474576271187$e$(-1.55504366571219e-16*t)]]
for the values above, the flipping time is $t_{1}:= 1.043009708$, where $\lambda_{0}$ from $3$ becomes $1$ and $\mu_{1}$ from $2$ becomes $5$.\\\\
The updated gamble at time $t_{1}:= 1.043009708$ is
\begin{equation*} 
 f_{t_1}= 
 \begin{array}{|c|}
  0.51463\\
  0.51463\\
  0.49872\\
  0.42535  
 \end{array}
 \end{equation*}\\\\\\
% t1:= 1.043009708, f(0) and f(1), hence l0 and m1 change
% updated gamble:=
% Matrix(4, 1, {(1, 1) = `+`(.5146315653, `-`(`*`(0.5469974431e-12\
%  , `*`(I)))), (2, 1) = `+`(.5146315645, `*`(0.5449909250e-12, 
%    `*`(I))), (3, 1) = `+`(.4987284326, `-`(`*`(0.9650839408e-12,\
%    `*`(I)))), (4, 1) = `+`(.4253503926, `*`(0.1269082515e-12, 
%    `*`(I)))})


--- Second Iteration ---

The maximum, row-wise, vector is the $f_{t_1}$ from \eqref{eq:ind_s1_4_f1_u} associated with generator matrix $\overline{Q}_{0}=Q_{0c}$.
Now the generator matrix of the process is
\begin{equation*} 
 \overline{Q}_{1}= 
 \begin{array}{|rrrr|}
  -3 & 3 & 0 & 0 \\
  2 & -5 & 3 & 0 \\
  0 & 2 & -3 & 1 \\
  0 & 0 & 5 & -5 
 \end{array}
 \end{equation*}
 % (from Q0c) Q1:= Matrix([[-3.0, 3.0, 0., 0.], [2.0, -5.0, 3.0, 0.], [0., 2.0, -3.0, 1.0], [0., 0., 5.0, -5.0]])
according to the changes in $\lambda_{1}$ and $\mu_{2}$, at flipping time point $t_{1}\coloneqq 0.910933094$.
Generator matrix $\overline{Q}_1$ is diagonalisable and yields 
\begin{equation*} 
 e^{\underline{Q}_{1}t}f_{t_1}= 
 \begin{array}{|r|}
 0.525043-0.00008e^{-8t}-0.00280e^{-5.73205t}-0.00221e^{-2.26794t}\\
 0.525043+0.00013e^{-8t}+0.00255e^{-5.73205t}-0.00053e^{-2.26794t}\\
 0.525043-0.00008e^{-8t}+0.00124e^{-5.73205t}+0.00098e^{-2.26794t}\\
 0.525043+0.00013e^{-8t}-0.00850e^{-5.73205t}+0.00179e^{-2.26794t}  
 \end{array}
 \end{equation*}
% eQ1tf := Matrix(4, 1, {(1, 1) = .5250432444-0.807774e-4*exp(-8.*t)-0.28032829e-2*exp(-5.732050808*t)-0.22113152e-2*exp(-2.267949192*t), (2, 1) = 0.1346288e-3*exp(-8.*t)+.5250432444+0.255290329e-2*exp(-5.732050808*t)-0.53959833e-3*exp(-2.267949192*t), (3, 1) = 0.124590353e-2*exp(-5.732050808*t)-0.807773e-4*exp(-8.*t)+.5250432444+0.98280660e-3*exp(-2.267949192*t), (4, 1) = 0.179866133e-2*exp(-2.267949192*t)-0.85096779e-2*exp(-5.732050808*t)+0.1346289e-3*exp(-8.*t)+.5250432444})
For the matrix above there the flipping time is $t_{2}=0.715044466$, where $\lambda_{2}$ from 1 becomes 3 and $\mu_{3}$ from 5 becomes 2.
The updated gamble is
\begin{equation*} 
 f_{t_2}= 
 \begin{array}{|c|}
  0.524559\\
  0.524979\\
  0.525257\\
  0.525257  
 \end{array}
 \end{equation*}
% t2:= 0.71504446605, , f(n-1) and f(n), hence change in l2 and m3
% updated Matrix(4, 1, {(1, 1) = .5245595801, (2, 1) = .5249794444, (3, 1) = .5252578238, (4, 1) = .5252578238})
\\\\

--- Third Iteration ---

According to the change in the second iteration, the generator matrix of the process is now
\begin{equation*} 
 \overline{Q}_{2}= 
 \begin{array}{|rrrr|}
  -3 & 3 & 0 & 0 \\
  2 & -5 & 3 & 0 \\
  0 & 2 & -5 & 3 \\
  0 & 0 & 2 & -2 
 \end{array}
 \end{equation*}
 % Matrix([[-3.0, 3.0, 0., 0.], [2, -5.0, 3.0, 0.], [0., 2.0, -5.0, 3.0], [0., 0., 2.0, -2.0]])
which is diagonalisable and yields 
\begin{equation*} 
 e^{\overline{Q}_{2}t}f_{t_2}= 
 \begin{array}{|r|}
 0.52512+0.000005e^{-8.46410t}-0.00012e^{-5t}-0.00044e^{-1.53589t}\\
 0.52512-0.000010e^{-8.46410t}+0.00008e^{-5t}-0.00021e^{-1.53589t}\\
 0.52512+0.000007e^{-8.46410t}+0.00008e^{-5t}+0.00004e^{-1.53589t}\\
 0.52512-0.000002e^{-8.46410t}-0.00005e^{-5t}+0.00019e^{-1.53589t}  
 \end{array}
 \end{equation*}
% eQ1tf := Matrix(4, 1, {(1, 1) = .5251204931+0.55208e-5*exp(-8.464101615*t)-0.1259596e-3*exp(-5.*t)-0.4404738e-3*exp(-1.535898385*t), (2, 1) = -0.100554e-4*exp(-8.464101615*t)+.5251204929+0.839732e-4*exp(-5.*t)-0.21496611e-3*exp(-1.535898385*t), (3, 1) = 0.839730e-4*exp(-5.*t)+0.79300e-5*exp(-8.464101615*t)+.5251204930+0.4542711e-4*exp(-1.535898385*t), (4, 1) = 0.19576631e-3*exp(-1.535898385*t)-0.5598204e-4*exp(-5.*t)-0.245342e-5*exp(-8.464101615*t)+.5251204929})
For the matrix above there is no flipping time and by taking $t\to\infty$, we have that
$\overline{E}(X=1) = 0.525120493$.
%no other flip, up prob = 0.525120493
\\\\\\


--- \textbf{State 2} ($f\coloneqq[0,0,1,0]^T$)---\\

***Lower probability ($\underline{E}(X=2)$)***\\

--- First Iteration ---\\
Again we encounter the problem that we do not know directly which generator matrix $Q$ minimises $Qf$.
We know that the form of the initial generator matrix should be
\begin{equation*} 
 \underline{Q}_{0}= 
 \begin{array}{|rrrr|}
  ? & ? & 0 & 0 \\
  ? & ? & 1 & 0 \\
  0 & 5 & -8 & 3 \\
  0 & 0 & 2 & -2 
 \end{array}
 \end{equation*}
 where for the unknown values we test the possible combinations of extreme points $\{\underline{\lambda},\overline{\lambda}\}\times\{\underline{\mu},\overline{\mu}\}$.\\\\\\
We start with the generator matrix 
\begin{equation*} 
 Q_{0a}= 
 \begin{array}{|rrrr|}
  -1 & 1 & 0 & 0 \\
  5 & -6 & 1 & 0 \\
  0 & 5 & -8 & 3 \\
  0 & 0 & 2 & -2 
 \end{array}
 \end{equation*}
% Q0a := Matrix([[-1.0, 1.0, 0., 0.], [5.0, -6.0, 1.0, 0.], [0., 5.0, -8.0, 3.0], [0., 0., 2.0, -2.0]])
which is diagonalisable and yields the following
\begin{equation*} 
 e^{Q_{0a}t}f= 
 \begin{array}{|r|}
  0.03076+0.02032e^{-10.13248t}-0.03211e^{-5.7523t}-0.01897e^{-1.1152t}\\
  0.03076-0.18557e^{-10.13248t}+0.15262e^{-5.7523t}+ 0.00218e^{-1.1152t}  \\
  0.03076+0.66529e^{-10.13248t}+ 0.19838e^{-5.7523t}+ 0.10554e^{-1.1152t}  \\
  0.03076-0.16361e^{-10.13248t}- 0.10573e^{-5.7523t}+ 0.23858e^{-1.1152t}  
 \end{array}
 \end{equation*}
% eQ0tf:=[[ 0.0203207627667553$e$(-10.1324858863621*t) - 0.0321157428973881$e$(-5.75230775846046*t) - 0.018974250638598$e$(-1.11520635517743*t) + 0.0307692307692308$e$(3.87907619831211e-16*t)]
%  [ -0.185579079167505$e$(-10.1324858863621*t) + 0.152623894139978$e$(-5.75230775846046*t) + 0.00218595425829589$e$(-1.11520635517743*t) + 0.0307692307692308$e$(3.87907619831211e-16*t)]
%  [ 0.665299111630016$e$(-10.1324858863621*t) + 0.198382468938967$e$(-5.75230775846046*t) + 0.105549188661786$e$(-1.11520635517743*t) + 0.0307692307692308$e$(3.87907619831211e-16*t)]
%  [ -0.163615189974248$e$(-10.1324858863621*t) - 0.105738911469438$e$(-5.75230775846046*t) + 0.238584870674455$e$(-1.11520635517743*t) + 0.0307692307692308$e$(3.87907619831211e-16*t)]]
for the values above, the flipping time is $t_{1}:= 0.2977232736$, where $\lambda_{2}$ from $3$ becomes $1$ and $\mu_{3}$ from $2$ becomes $5$.\\\\
The updated gamble at time $t_{1}:= 0.2977232736$ is
\begin{equation} \label{eq:ind_s2_4_f1} 
 f_{t_1}= 
 \begin{array}{|c|}
  0.012357\\
  0.050783\\
  0.174860\\
  0.174860  
 \end{array}
 \end{equation}\\\\\\
% :=Matrix(4, 1, {(1, 1) = 0.1235715678e-1-0.1723169800e-14*I, (2, 1) = 0.5078383445e-1-0.1188476322e-13*I, (3, 1) = .1748607622+0.1435443935e-12*I, (4, 1) = .1748607636-0.2736088576e-13*I})
We take now generator matrix 
\begin{equation*} 
 Q_{0b}= 
 \begin{array}{|rrrr|}
  -1 & 1 & 0 & 0 \\
  2 & -3 & 1 & 0 \\
  0 & 5 & -8 & 3 \\
  0 & 0 & 2 & -2 
 \end{array}
 \end{equation*}
 which is also diagonalisable and yields
 % Q0b:= Matrix([[-1.0, 1.0, 0., 0.], [2.0, -3.0, 1.0, 0.], [0., 5.0, -8.0, 3.0], [0., 0., 2.0, -2.0]])
\begin{equation*} 
 e^{Q_{0b}t}f= 
 \begin{array}{|r|}
  0.05714+0.01490e^{-9.57946t}-0.02861e^{-3.32005t}-0.04343e^{-1.10047t}\\
  0.05714-0.12790e^{-9.57946t}+ 0.06639e^{-3.32005t}+ 0.00436e^{-1.10047t}\\
  0.05714+0.81171e^{-9.57946t}+ 0.03598e^{-3.32005t}+ 0.09515e^{-1.10047t}\\
  0.05714-0.21418e^{-9.57946t}- 0.05452e^{-3.32005t}+ 0.21156e^{-1.10047t}  
 \end{array}
 \end{equation*}
 % eQ0tf:=[[ 0.0149080134198346$e$(-9.57946662987692*t) - 0.0286182489743913$e$(-3.32005527954441*t) - 0.0434326215883004$e$(-1.10047809057866*t) + 0.057142857142857$e$(-3.73040594638487e-16*t)]
 % [ -0.127902803653228$e$(-9.57946662987692*t) + 0.066395919624353$e$(-3.32005527954441*t) + 0.00436402688601787$e$(-1.10047809057866*t) + 0.0571428571428571$e$(-3.73040594638487e-16*t)]
 % [ 0.811716201664445$e$(-9.57946662987692*t) + 0.0359861333328019$e$(-3.32005527954441*t) + 0.0951548078598956$e$(-1.10047809057866*t) + 0.0571428571428571$e$(-3.73040594638487e-16*t)]
 % [ -0.2141882117311$e$(-9.57946662987692*t) - 0.054522161140436$e$(-3.32005527954441*t) + 0.211567515728679$e$(-1.10047809057866*t) + 0.057142857142857$e$(-3.73040594638487e-16*t)]]]
for the values above, the flipping time is $t_{1}:= 0.3143651106$, where $\lambda_{2}$ from $3$ becomes $1$ and $\mu_{3}$ from $2$ becomes $5$.\\\\
The updated gamble at time $t_{1}:= 0.3143651106$ is
\begin{equation*} 
 f_{t_1}= 
 \begin{array}{|c|}
  0.017068\\
  0.077316\\
  0.177094\\
  0.177094  
 \end{array}
 \end{equation*}\\\\\\
 % t1:=0.3143651106 f(n-1) and f(n), hence l2 and m3 change
 % updated gamble:=Matrix(4, 1, {(1, 1) = 0.1706823314e-1-0.5533610600e-14*I, (2, 1) = 0.7731625509e-1+0.2945625930e-13*I, (3, 1) = .1770945568-0.1700169844e-12*I, (4, 1) = .1770945576+0.5204785890e-13*I})
We take now generator matrix 
\begin{equation*} 
 Q_{0c}= 
 \begin{array}{|rrrr|}
  -3 & 3 & 0 & 0 \\
  5 & -6 & 1 & 0 \\
  0 & 5 & -8 & 3 \\
  0 & 0 & 2 & -2 
 \end{array}
 \end{equation*}
 which is also diagonalisable and yields
 % Q0c := Matrix([[-3.0, 3.0, 0., 0.], [5.0, -6.0, 1.0, 0.], [0., 5.0, -8.0, 3.0], [0., 0., 2.0, -2.0]])
\begin{equation*} 
 e^{Q_{0c}t}f= 
 \begin{array}{|r|}
  0.06315+0.075358e^{-10.60074t}- 0.106118e^{-7.14500t}- 0.032397e^{-1.25425t}\\
  0.06315-0.190926e^{-10.60074t}+ 0.146621e^{-7.14500t}- 0.018852e^{-1.25425t}\\
  0.06315+0.501612e^{-10.60074t}+ 0.362712e^{-7.14500t}+ 0.072517e^{-1.25425t}\\
  0.06315-0.116643e^{-10.60074t}- 0.140996e^{-7.14500t}+ 0.194482e^{-1.25425t}  
 \end{array}
 \end{equation*}
%  eQ0tf:=[[ 0.0753583618246389$e$(-10.6007432753446*t) - 0.106118925015431$e$(-7.1450048447915*t) - 0.0323973315460502$e$(-1.2542518798639*t) + 0.0631578947368422$e$(-8.42385307333489e-17*t)]
%  [ -0.190926520626536$e$(-10.6007432753446*t) + 0.146621152771009$e$(-7.1450048447915*t) - 0.0188525268813144$e$(-1.2542518798639*t) + 0.0631578947368422$e$(-8.42385307333489e-17*t)]
%  [ 0.501612096734285$e$(-10.6007432753446*t) + 0.362712694805435$e$(-7.1450048447915*t) + 0.0725173137234378$e$(-1.2542518798639*t) + 0.0631578947368422$e$(-8.42385307333489e-17*t)]
%  [ -0.116643894760174$e$(-10.6007432753446*t) - 0.140996055687926$e$(-7.1450048447915*t) + 0.194482055711258$e$(-1.2542518798639*t) + 0.0631578947368421$e$(-8.42385307333489e-17*t)]]
% ()
for the values above, the flipping time is $t_{1}:= 0.3017854416$, where $\lambda_{2}$ from $3$ becomes $1$ and $\mu_{3}$ from $2$ becomes $5$.\\\\
The updated gamble at time $t_{1}:= 0.3017854416$ is
\begin{equation*} 
 f_{t_1}= 
 \begin{array}{|c|}
  0.031760\\
  0.059429\\
  0.175274\\
  0.175274  
 \end{array}
 \end{equation*}\\\\\\
 % t1:=0.3017854416 f(n-1) and f(n), hence l2 and m3 change
 % updated gamble:=Matrix(4, 1, {(1, 1) = 0.3176001555e-1-0.7006432300e-14*I, (2, 1) = 0.5942950207e-1+0.1169321260e-13*I, (3, 1) = .1752742783-0.1248065147e-13*I, (4, 1) = .1752742772+0.8267682100e-14*I})
 The last generator matrix we have to consider is
\begin{equation*} 
 Q_{0d}= 
 \begin{array}{|rrrr|}
  -3 & 3 & 0 & 0 \\
  2 & -3 & 1 & 0 \\
  0 & 5 & -8 & 3 \\
  0 & 0 & 2 & -2 
 \end{array}
 \end{equation*}
 which is also diagonalisable and yields
 % Q0c := Matrix([[-3.0, 3.0, 0., 0.], [2.0, -3.0, 1.0, 0.], [0., 5.0, -8.0, 3.0], [0., 0., 2.0, -2.0]])
\begin{equation*} 
 e^{Q_{0d}t}f= 
 \begin{array}{|r|}
 0.09230+0.061532e^{-9.65331t}- 0.105882e^{-5t}- 0.047958e^{-1.34668t}\\
  0.09230-0.136465e^{-9.65331t}+ 0.070588e^{-5t}- 0.026429e^{-1.34668t}\\
  0.09230+0.784884e^{-9.65331t}+ 0.070588e^{-5t}+ 0.092307e^{-1.34668t}\\
  0.09230-0.205109e^{-9.65331t}- 0.047058e^{-5t}+ 0.159860e^{-1.34668t}  
 \end{array}
 \end{equation*}
%  eQ0tf:=[[ 0.0615329331905876$e$(-9.65331193145904*t) - 0.105882352941177$e$(-5.0*t) - 0.0479582725571032$e$(-1.34668806854096*t) + 0.0923076923076922$e$(1.37435128030003e-16*t)]
%  [ -0.136465932858202$e$(-9.65331193145904*t) + 0.0705882352941178$e$(-5.0*t) - 0.0264299947436077$e$(-1.34668806854096*t) + 0.0923076923076922$e$(1.37435128030003e-16*t)]
%  [ 0.78488455294199$e$(-9.65331193145904*t) + 0.0705882352941179$e$(-5.0*t) + 0.0522195194562001$e$(-1.34668806854096*t) + 0.0923076923076922$e$(1.37435128030003e-16*t)]
%  [ -0.205109777301958$e$(-9.65331193145904*t) - 0.0470588235294119$e$(-5.0*t) + 0.159860908523677$e$(-1.34668806854096*t) + 0.0923076923076922$e$(1.37435128030003e-16*t)]]
% ()
for the values above, the flipping time is $t_{1}:= 0.3176058928$, where $\lambda_{2}$ from $3$ becomes $1$ and $\mu_{3}$ from $2$ becomes $5$.\\\\
The updated gamble at time $t_{1}:= 0.3176058928$ is
\begin{equation*} 
 f_{t_1}= 
 \begin{array}{|c|}
  0.024765\\
  0.059429\\
  0.177360\\
  0.177360  
 \end{array}
 \end{equation*}\\\\\\
% t1:=0.3176058928 f(n-1) and f(n), hence l2 and m3 change

% updated gamble:=Matrix(4, 1, {(1, 1) = 0.4227237769e-1, (2, 1) = 0.8313794788e-1, (3, 1) = .1773609346, (4, 1) = .1773609344})

--- Second Iteration ---\\
The minimum, row-wise, vector is the $f_{t_1}$ from \eqref{eq:ind_s2_4_f1} associated with generator matrix $Q_{0a}$.
Now the generator matrix of the process is
\begin{equation*} 
 \underline{Q}_{1}= 
 \begin{array}{|rrrr|}
  -1 & 1 & 0 & 0 \\
  5 & -6 & 1 & 0 \\
  0 & 5 & -6 & 1 \\
  0 & 0 & 5 & -5 
 \end{array}
 \end{equation*}
according to the changes in $\lambda_{2}$ and $\mu_{3}$, at flipping time point $t_{1}\coloneqq 0.2977232736$.
Generator matrix $\underline{Q}_1$ is diagonalisable and yields 
\begin{equation*} 
 e^{\underline{Q}_{1}t}f_{t_1}= 
 \begin{array}{|r|}
 0.0247654+0.001092e^{-9.16227t}-0.003202e^{-6t}-0.010298e^{-2.83772t}\\
 0.0247654-0.008918e^{-9.16227t}+0.016011e^{-6t}+0.018926e^{-2.83772t}\\
 0.0247654+0.022740e^{-9.16227t}+0.016011e^{-6t}-0.024765e^{-2.83772t}\\
 0.0247654+0.027317e^{-9.16227t}-0.080055e^{-6t}-0.257468e^{-2.83772t}  
 \end{array}
 \end{equation*}
% eQ1tf:= Matrix(4, 1, {(1, 1) = 0.2476541689e-1+0.1092699408e-2*exp(-9.162277660*t)-0.3202223119e-2*exp(-6.*t)-0.1029873639e-1*exp(-2.837722340*t), (2, 1) = -0.891891595e-2*exp(-9.162277660*t)+0.2476541689e-1+0.1601111558e-1*exp(-6.*t)+0.1892621796e-1*exp(-2.837722340*t), (3, 1) = 0.1601111565e-1*exp(-6.*t)+0.2274059161e-1*exp(-9.162277660*t)+0.2476541677e-1+.1113436381*exp(-2.837722340*t), (4, 1) = .2574684098*exp(-2.837722340*t)-0.8005557778e-1*exp(-6.*t)-0.2731748530e-1*exp(-9.162277660*t)+0.2476541679e-1})
For the matrix above there is no flipping time and by taking $t\to\infty$, we have that
$\underline{E}(X=2) = 0.0247654$.\\\\\\

***Upper probability ($\overline{E}(X=2)$)***\\

--- First Iteration ---

Again we encounter the problem that we do not know directly which generator matrix $Q$ minimises $Qf$.
We know that the form of the initial generator matrix should be
\begin{equation*} 
 \overline{Q}_{0}= 
 \begin{array}{|rrrr|}
  ? & ? & 0 & 0 \\
  ? & ? & 3 & 0 \\
  0 & 2 & -3 & 1 \\
  0 & 0 & 5 & -5 
 \end{array}
 \end{equation*}\\\\
We start with the generator matrix 
\begin{equation*} 
 Q_{0a}= 
 \begin{array}{|rrrr|}
  -1 & 1 & 0 & 0 \\
  5 & -8 & 3 & 0 \\
  0 & 2 & -3 & 1 \\
  0 & 0 & 5 & -5 
 \end{array}
 \end{equation*}
% Q0a:= Matrix([[-1.0, 1.0, 0., 0.], [5.0, -8.0, 3.0, 0.], [0., 2.0, -3.0, 1.0], [0., 0., 5.0, -5.0]])
which is diagonalisable and yields the following
\begin{equation*} 
 e^{Q_{0a}t}f= 
 \begin{array}{|r|}
  0.19230+0.04765e^{-9.65331t}- 0.02941e^{-6t}- 0.21054e^{-1.34668t}\\
  0.19230-0.41236e^{-9.65331t}+ 0.14705e^{-6t}+ 0.07299e^{-1.34668t}\\
  0.19230+0.14783e^{-9.65331t}+ 0.14705e^{-6t}+ 0.51280e^{-1.34668t}\\
  0.19230-0.15884e^{-9.65331t}- 0.73529e^{-6t}+ 0.70183e^{-1.34668t}  
 \end{array}
 \end{equation*}
% eQ0tf:= [[ 0.0476536071553272$e$(-9.65331193145905*t) - 0.0294117647058824$e$(-6.0*t) - 0.210549534757137$e$(-1.34668806854096*t) + 0.192307692307692$e$(2.22044604925031e-16*t)]
 % [ -0.412361527374254$e$(-9.65331193145905*t) + 0.147058823529412$e$(-6.0*t) + 0.0729950115371501$e$(-1.34668806854096*t) + 0.192307692307693$e$(2.22044604925031e-16*t)]
 % [ 0.14783139916863$e$(-9.65331193145905*t) + 0.147058823529412$e$(-6.0*t) + 0.512802084994265$e$(-1.34668806854096*t) + 0.192307692307692$e$(2.22044604925031e-16*t)]
 % [ -0.158845357184424$e$(-9.65331193145905*t) - 0.735294117647059$e$(-6.0*t) + 0.70183178252379$e$(-1.34668806854096*t) + 0.192307692307692$e$(2.22044604925031e-16*t)]]
for the values above, the flipping time is $t_{1}:= 0.3508833687$, where $\lambda_{2}$ from $1$ becomes $3$ and $\mu_{3}$ from $5$ becomes $2$.\\\\
The updated gamble at time $t_{1}:= 0.3508833687$ is
\begin{equation*}  
 f_{t_1}= 
 \begin{array}{|c|}
  0.059074\\
  0.241788\\
  0.534910\\
  0.534910  
 \end{array}
 \end{equation*}\\\\\\
% :=Matrix(4, 1, {(1, 1) = 0.590745710e-1, (2, 1) = .2417882211, (3, 1) = .5349105139, (4, 1) = .5349105140})
We consider now the generator matrix 
\begin{equation*} 
 Q_{0b}= 
 \begin{array}{|rrrr|}
  -1 & 1 & 0 & 0 \\
  2 & -5 & 3 & 0 \\
  0 & 2 & -3 & 1 \\
  0 & 0 & 5 & -5 
 \end{array}
 \end{equation*}
% Q0b := Matrix([[-1.0, 1.0, 0., 0.], [2.0, -5.0, 3.0, 0.], [0., 2.0, -3.0, 1.0], [0., 0., 5.0, -5.0]]);
which is diagonalisable and yields the following
\begin{equation*} 
 e^{Q_{0b}t}f= 
 \begin{array}{|r|}
  0.3125+0.06761e^{-7.57802t}- 0.01248e^{-5.20506t}- 0.36762e^{-1.21691t}\\
  0.3125-0.44475e^{-7.57802t}+ 0.05251e^{-5.20506t}+ 0.07974e^{-1.21691t}\\
  0.3125+0.33712e^{-7.57802t}+ 0.00473e^{-5.20506t}+ 0.34563e^{-1.21691t}\\
  0.3125-0.65384e^{-7.57802t}- 0.11547e^{-5.20506t}+ 0.45682e^{-1.21691t}  
 \end{array}
 \end{equation*}
% eQ0tf:= [[ 0.0676129513915471$e$(-7.57802063330344*t) - 0.0124890885681177$e$(-5.20506763920773*t) - 0.36762386282343$e$(-1.21691172748883*t) + 0.3125$e$(2.22044604925031e-16*t)]
 % [ -0.444759389332139$e$(-7.57802063330344*t) + 0.052517462180991$e$(-5.20506763920773*t) + 0.079741927151148$e$(-1.21691172748883*t) + 0.3125$e$(2.22044604925031e-16*t)]
 % [ 0.337124326590199$e$(-7.57802063330344*t) + 0.00473618171653271$e$(-5.20506763920773*t) + 0.345639491693269$e$(-1.21691172748883*t) + 0.3125$e$(2.22044604925031e-16*t)]
 % [ -0.653843344454177$e$(-7.57802063330344*t) - 0.115478525398516$e$(-5.20506763920773*t) + 0.456821869852693$e$(-1.21691172748883*t) + 0.3125$e$(2.22044604925031e-16*t)]]
for the values above, the flipping time is $t_{1}:= 0.3854708723$, where $\lambda_{2}$ from $1$ becomes $3$ and $\mu_{3}$ from $5$ becomes $2$.\\\\
The updated gamble at time $t_{1}:= 0.3854708723$ is
\begin{equation*} 
 f_{t_1}= 
 \begin{array}{|c|}
  0.08448\\
  0.34548\\
  0.54752\\
  0.54752  
 \end{array}
 \end{equation*}\\\\\\
% updated gamble := Matrix(4, 1, {(1, 1) = 0.8448685175e-1+0.1202081030e-13*I, (2, 1) = .3454856735-0.8199224400e-13*I, (3, 1) = .5475225685+0.6653703580e-13*I, (4, 1) = .5475225685-0.1395164369e-12*I})
We consider now the generator matrix 
\begin{equation*} 
 Q_{0c}= 
 \begin{array}{|rrrr|}
  -3 & 3 & 0 & 0 \\
  5 & -8 & 3 & 0 \\
  0 & 2 & -3 & 1 \\
  0 & 0 & 5 & -5 
 \end{array}
 \end{equation*}
% Q0c := Matrix([[-3.0, 3.0, 0., 0.], [5.0, -8.0, 3.0, 0.], [0., 2.0, -3.0, 1.0], [0., 0., 5.0, -5.0]]);
which is diagonalisable and yields the following
\begin{equation*} 
 e^{Q_{0c}t}f= 
 \begin{array}{|r|}
  0.33582+0.11988e^{-10.7912t}- 0.09084e^{-6.20871t}- 0.36486e^{-2t}\\
  0.33582-0.31136e^{-10.7912t}+ 0.09716e^{-6.20871t}- 0.12162e^{-2t}\\
  0.33582+ 0.08988e^{-10.7912t}+ 0.20942e^{-6.20871t} + 0.36486e^{-2t}\\
  0.33582-0.07760e^{-10.7912t}- 0.86632e^{-6.20871t}+ 0.60810e^{-2t}  
 \end{array}
 \end{equation*}
% [[ 0.119889800166391$e$(-10.7912878474779*t) - 0.090845830823914$e$(-6.20871215252208*t) - 0.364864864864865$e$(-2.0*t) + 0.335820895522388$e$(-8.52182907573606e-16*t)]
% [ -0.311365314357653$e$(-10.7912878474779*t) + 0.0971660404568862$e$(-6.20871215252208*t) - 0.121621621621621$e$(-2.0*t) + 0.335820895522388$e$(-8.52182907573606e-16*t)]
% [ 0.0898870724202348$e$(-10.7912878474779*t) + 0.209427167192513$e$(-6.20871215252208*t) + 0.364864864864865$e$(-2.0*t) + 0.335820895522388$e$(-8.52182907573606e-16*t)]
% [ -0.0776054262778356$e$(-10.7912878474779*t) - 0.86632357735266$e$(-6.20871215252208*t) + 0.608108108108108$e$(-2.0*t) + 0.335820895522388$e$(-8.52182907573606e-16*t)]]
for the values above, the flipping time is $t_{1}:= 0.3602411839$, where $\lambda_{2}$ from $1$ becomes $3$ and $\mu_{3}$ from $5$ becomes $2$.\\\\
The updated gamble at time $t_{1}:= 0.3602411839$ is
\begin{equation*} 
 f_{t_1}= 
 \begin{array}{|c|}
  0.15106\\
  0.28064\\
  0.53754\\
  0.53754  
 \end{array}
 \end{equation*}\\\\\\
% Matrix(4, 1, {(1, 1) = .1510612321, (2, 1) = .2806467535, (3, 1) = .5375469587, (4, 1) = .5375469567})
We consider now the generator matrix 
\begin{equation*} 
 Q_{0d}= 
 \begin{array}{|rrrr|}
  -3 & 3 & 0 & 0 \\
  2 & -5 & 3 & 0 \\
  0 & 2 & -3 & 1 \\
  0 & 0 & 5 & -5 
 \end{array}
 \end{equation*}
% Q0d := Matrix([[-3.0, 3.0, 0., 0.], [2.0, -5.0, 3.0, 0.], [0., 2.0, -3.0, 1.0], [0., 0., 5.0, -5.0]]);
which is diagonalisable and yields the following
\begin{equation*} 
 e^{Q_{0d}t}f= 
 \begin{array}{|r|}
  0.43269+0.259618e^{-8t}- 0.14630e^{-5.73205t}- 0.54600e^{-2.26794t}\\
  0.43269-0.43269e^{-8t}+ 0.13323e^{-5.73205t}- 0.13323e^{-2.26794t}\\
  0.43269+ 0.25961e^{-8t}+ 0.06502e^{-5.73205t} + 0.242669e^{-2.26794t}\\
  0.43269--0.43269e^{-8t}- 0.44411e^{-5.73205t}+ 0.44411e^{-2.26794t}  
 \end{array}
 \end{equation*}
% eQ0tf:= [[ 0.259615384615384$e$(-7.99999999999999*t) - 0.146301829895899$e$(-5.73205080756888*t) - 0.546005862411793$e$(-2.26794919243112*t) + 0.432692307692307$e$(4.44089209850063e-16*t)]
% [ -0.432692307692307$e$(-7.99999999999999*t) + 0.133234677505298$e$(-5.73205080756888*t) - 0.133234677505298$e$(-2.26794919243112*t) + 0.432692307692307$e$(4.44089209850063e-16*t)]
% [ 0.259615384615385$e$(-7.99999999999999*t) + 0.0650230355092883$e$(-5.73205080756888*t) + 0.242669272183019$e$(-2.26794919243112*t) + 0.432692307692308$e$(4.44089209850063e-16*t)]
% [ -0.432692307692308$e$(-7.99999999999999*t) - 0.444115591684327$e$(-5.73205080756888*t) + 0.444115591684327$e$(-2.26794919243112*t) + 0.432692307692308$e$(4.44089209850063e-16*t)]]
for the values above, the flipping time is $t_{1}:= 0.3951040505$, where $\lambda_{2}$ from $1$ becomes $3$ and $\mu_{3}$ from $5$ becomes $2$.\\\\
The updated gamble at time $t_{1}:= 0.3951040505$ is
\begin{equation*} 
 f_{t_1}= 
 \begin{array}{|c|}
  0.20564\\
  0.37380\\
  0.54950\\
  0.54950  
 \end{array}
 \end{equation*}
 which also the largest among the multiple $f_{t_1}$ that we found.\\\\\\
% updated gamble := Matrix(4, 1, {(1, 1) = .2056410598, (2, 1) = .3738052614, (3, 1) = .5495003882, (4, 1) = .5495003887})


--- Second Iteration ---

The maximum, row-wise, vector is the aforementioned $f_{t_1}$ associated with generator matrix $\overline{Q}_{0}=Q_{0d}$.
Now the generator matrix of the process is
\begin{equation*} 
 \overline{Q}_{1}= 
 \begin{array}{|rrrr|}
  -3 & 3 & 0 & 0 \\
  2 & -5 & 3 & 0 \\
  0 & 2 & -5 & 3 \\
  0 & 0 & 2 & -2 
 \end{array}
 \end{equation*}
 % (from Q0d) Q1:= Matrix([[-3.0, 3.0, 0., 0.], [2.0, -5.0, 3.0, 0.], [0., 2.0, -5.0, 3.0], [0., 0., 2.0, -2.0]]);
according to the changes in $\lambda_{2}$ and $\mu_{3}$, at flipping time point $t_{1}\coloneqq 0.3951040505$.
Generator matrix $\overline{Q}_1$ is diagonalisable and yields 
\begin{equation*} 
 e^{\overline{Q}_{1}t}f_{t_1}= 
 \begin{array}{|r|}
 0.474743+0.012064e^{-8.46410t}-0.050449e^{-5t}-0.230716e^{-1.53589t}\\
 0.474743-0.021973e^{-8.46410t}+0.033632e^{-5t}+0.018926e^{-1.53589t}\\
 0.474743+0.017329e^{-8.46410t}+0.033632e^{-5t}-0.023794e^{-1.53589t}\\
 0.474743-0.005361e^{-8.46410t}-0.022421e^{-5t}+0.102540e^{-1.53589t}  
 \end{array}
 \end{equation*}
% eQ1tf := Matrix(4, 1, {(1, 1) = .4747432170+0.1206406983e-1*exp(-8.464101615*t)-0.504492605e-1*exp(-5.*t)-.2307169665*exp(-1.535898385*t), (2, 1) = -0.219731014e-1*exp(-8.464101615*t)+.4747432170+0.3363284033e-1*exp(-5.*t)-.1125976941*exp(-1.535898385*t), (3, 1) = 0.3363284019e-1*exp(-5.*t)+0.173296384e-1*exp(-8.464101615*t)+.4747432168+0.2379469202e-1*exp(-1.535898385*t), (4, 1) = .1025408740*exp(-1.535898385*t)-0.2242189348e-1*exp(-5.*t)-0.536180862e-2*exp(-8.464101615*t)+.4747432168})
For the matrix above there is no flipping time and by taking $t\to\infty$, we have that
$\overline{E}(X=2) = 0.474743217$.\\\\\\


% Lower probability = 0.03225806452

% No flipping time--- Matrix([[-1, 1, 0], [5, -6, 1], [0, 5, -5]]), $\lambda_{i}=\underline{\lambda}=1$ and $\mu_{i}=\overline{\mu}=5$\\

% Upper probability = 0.4736841978 

% No flipping time--- Matrix([[-3, 3, 0], [2, -5, 3], [0, 2, -2]]), i.e $\lambda_{i}=\overline{\lambda}=3$ and $\mu_{i}=\underline{\mu}=2$


% $f=[1,8,5]^T$

% $\stateset:=\{0,1,2,3\}$, $\lambda \in[1,3]$ and $\mu \in [2,5]$

% $f=[10,2,5,7]^T$ 
% $f=[1,5,3,9]^T$ 
THE PART THAT FOLLOWS WILL BE MOVED IN ANOTHER PLACE\\\\\\

***The model

***Solution to the dif Eq.
\end{comment}

\bibliographystyle{plain}
\bibliography{general}




\end{document}
