\documentclass[twoside,11pt]{letter}
\usepackage{amsmath}
\usepackage{xcolor}
\usepackage{amssymb}
\usepackage{nicefrac}

\begin{document}

Dear Sir/Madam,

We would very much like to thank you for the time and effort that you took in reviewing our paper ``Imprecise Continuous-Time Markov Chains'' for the International Journal of Approximate Reasoning.

{\bf ******* En dan wel nog wat meer dank }

*** jasper

With kind regards,

Thomas Krak\\
Jasper De Bock\\
Arno Siebes
\newline\newline

TODO

*** In the end, we need to check whether some of the numbers of equations and/or results have changed, and mention this here. For example: ... the use of Equation (5)---which corresponds to (7) in the new version---... ***

Ik zou de toepassing ook nog eens vermelden in Sectie 6.3, ter compensatie van het voorbeeld, aangezien dat voorbeeld zeer grote verschillen (en dus nutteloze buitenbenaderingen) suggereert, wat in die telecommunicatietoepassing niet het geval bleek.


We moeten nog wat referenties toevoegen naar de recentste isipta papers...

\newpage
{\bf Area Editor}

{\bf
I am recommending a minor revision even though the extent of the revision could be major, depending on the authors. I think there are no serious issues about this paper other than length. I would discourage the authors to do lengthy papers like this in the future. It is very difficult to find reviewers and it requires a huge effort by them, which is not really fair. Moreover the length of the paper will discourage potential readers to get into it. As with this paper, I leave to the authors the decision about what to do with it, whether following the suggestions of R3 or not. In any case, I would recommend the authors to address all the comments by the reviewers, in particular those aimed at reducing some parts/make the reading easier.
}

*** jasper
add reply and explain what we have done

\newpage
{\bf Reviewer 1}

{\bf
\dots the authors developed most of the concepts from scratch.
This justifies the length of the manuscript, although, in my opinion at some parts it could be shortened 
by omitting some straightforward proofs or referring to the existing theory of 
precise continuous time Markov chains, especially in Sections 3 and 5.}

{\color{green}
We agree with the observation that some of the proofs/results in Section 5 are conceptually similar to known results from the literature. However, we are not aware of other work that uses full conditional probabilities and/or coherence to characterise (precise) continuous-time Markov chains. Furthermore, the mathematical equivalence between our characterisation and, say, the measure-theoretic one, does not seem straightforward to us (or even, indeed, \emph{exactly} true, since we do not require e.g. $\sigma$-additivity). This makes it difficult to refer back to known results, at least on a technical level. We have therefore decided to retain our proofs for the results in Section 5.

Regarding the more straightforward proofs for some of the results in Section 3, we agree that these could in principle be removed. However, those would then be the only (formal) statements in the paper that are given without a corresponding proof in the appendix, which we feel could lead to confusion. Since the proofs in question are only several lines long, adding a ``left-as-an-exercise'' comment to the main text would also not help (much) to reduce the overall length of the paper. We have therefore kept these proofs in the appendix for now, but are willing to remove them should you insist.
}

{\bf The authors claim that equation (57) is stronger than equation (56). 
This is clearly true if the operators would be defined on infinitely dimensional spaces, 
but in the finite dimensional space I do not see why these equations would not be equivalent, 
especially considering similar equivalence in Lemma F2. If this is in fact so, this claim should be additionally justified. }

{\color{green}
Our claim here is not that these two ways of interpreting the derivative do not lead to the same solution. In fact, in this case, as we show, they do. Our point is merely that Equation (57) trivially implies (56) and that, in that sense,  Equation (57) is stronger than (56). In order to try and clarify this, we have slightly reformulated the text.

Regarding the question of whether both types of derivatives are equivalent in general: we think this will depend on the properties of the operator $\underline{Q}$. If $\underline{Q}$ is a lower transition operator, then indeed, judging from Lemma F2, it seems plausible to assume that they are equivalent. However, since this equivalence is not needed here, we prefer to leave this as future work.
}


{\bf Lemma G3: In the proof, third line, it is suggested that the norm of Q(f) is smaller than the maximal norm of a 
vector that bounds it from above or below. This might be true in this case, but it does not just follow from such an inequality. 
Consider, for instance, $(-1, 0) \leq (-1, 1) \leq (0, 1)$, where the norm of the middle vector is greater than the norm of both bounding vectors. }

{\color{green}
The implication in the third line (i.e. that Equation (132) holds) follows from the fact that we use the \emph{maximum} norm, rather than e.g. the Euclidean norm.
Similarly, in the example you give, we have under the maximum norm that
\begin{equation}
\lVert(-1,0)\rVert = \lVert(-1,1)\rVert = \lVert(0,1)\rVert = 1,
\vspace{7pt}
\end{equation}
and hence also $\lVert(-1,1)\rVert \leq \max\{\lVert(-1,0)\rVert, \lVert(0,1)\rVert\}$, as per the statement.
}

{\bf Lemma G.4 could be replaced by an easy observation.   }

{\color{green}
We unfortunately do not really see an easier argument than the one that was given; note that we cannot directly apply Property LT5 to all $n$ steps simultaneously, nor is Equation (49) sufficient.

Regarding the direct use of Equation (49) to try to construct an easier argument, note that for functions $f \leq g$ it does not in general hold that also $\underline{Q}f \leq \underline{Q}g$. For example, consider $(-1, 0) \leq (0, 0)$, and the transition rate matrix $Q$ (which is trivially a lower transition rate operator $\underline{Q}$), given by
\begin{equation}
Q = \left[\begin{array}{rr}
-1 & 1 \\
0 & 0
\end{array}\right]
\end{equation}
for which it holds that
\begin{equation}
Q\left[\begin{array}{r}-1 \\ 0\end{array}\right] = \left[\begin{array}{r}
1 \\ 0\end{array}\right] > Q\left[\begin{array}{r}0 \\ 0\end{array}\right] = \left[\begin{array}{r}0 \\ 0\end{array}\right]\,.
\end{equation}

In any case, the induction argument that we give in Lemma G.4 is the simplest that we have been able to find. If you know of an easier argument, we would of course be willing to incorporate that instead.
}

{\bf The notion of well behavedness, as far as I understand, means that the corresponding transition systems satisfy the Lipschitz condition. 
This condition is very often applied in the theory of dynamic systems. 
The authors should check whether any implications of well behavedness coincide with known results based on Lipschitz condition.  }

*** thomas

We are actually not entirely sure if well-behavedness and Lipschitz-continuity coincide exactly, although it is clear that the latter implies the former.

**** DE ANDERE KANT OP IS NIET HELEMAAL TRIVIAAL. TEN EERSTE, WE EISEN HET ALLEEN OP LOKAAL NIVEAU, TEN TWEEDE EISEN WE NIET DAT DE ``LIPSCHITZ CONSTANTE'' ONAFHANKELIJK IS VAN $t$. 

*** Dat wil zeggen, de well-behavedness-eis kan je herschrijven als dat, voor alle $t$, er een $c_t\in R$ is zodanig dat, voor $\Delta$ klein genoeg,
\begin{equation}
\lvert P(X_{t+\Delta}\,\vert\,X_t) - P(X_{t}\,\vert\,X_t) \rvert < \Delta c_t
\end{equation}
Het eerste punt is dus dat we niks eisen voor `grote' $\Delta$. Het tweede punt is dat om Lipschitz-continue te zijn, we ook zouden moeten hebben dat
\begin{equation*}
\sup_{t} c_t < +\infty\,.
\end{equation*}
Of dat geimplicteerd is (samen met de definities van stochastische processen).... ik heb echt geen idee.

{\bf Why are some environments, e.g. examples definitions, enumerated differently than another? }

**** thomas

To answer the question, this was due to a stylistic preference of one of the authors. {\color{red}*** Van wie? Als ik het was, dan weet ik het alvast niet meer, en dan mag hier wat mij betreft gewoon staan dat we akkoord gaan. Dat mag van mij trouwens ook veranderen in onze `Arxiv' versie van de paper. ***} Regardless, the current version has environments enumerated as per the IJAR/Elsevier style.

\newpage
{\bf Reviewer 2}

{\bf \ldots Could some words be said about the case where different sets may be given for different time intervals (e.g., modelling the fact that, as time goes along, we are less and less certain what is our transition models)? Does it modify significantly the authors result, or do we have to simply apply the provided methods for each time-intervals with different models? Indeed, page 32, at the top, suddenly sets of rate matrices become time-independent, whereas the page before (equation 38), rate matrices were indeed time-dependent. I understand very well that rate matrices may be changing within the set of rate matrices, hence that equation (38) can somehow be included in sets of rate matrices, yet one wonder what would happen if sets of rate matrices were themselves time-dependent. }

{\color{green}
This is a very interesting question, and something that would definitely be worth looking into in future work. The technical subtlety lies in the kind of continuity assumptions that should be imposed on these changing parameter-sets. However, for time-dependent sets that are, say, piecewise-constant, we do indeed expect that our methods will translate fairly naturally to this more general setting. We now comment on this in our conclusions.
}

{\bf Similarly, the third part exclusively deals with generic stochastic process that are within the set of provided transition matrices. Some warnings should be sent to the reader about that (in particular practitioners), as well as some proper cautionary notes: while the given bounds certainly provide cautious bounds for the other cases, example 15 certainly suggests that these bounds may be quite conservative, and I guess the same may happen between W and WM in the case of multiple time points. In particular, stating at the end of P35 that �one of the main aims of this paper � for the different types of ICTMC�s �� seems to me quite an over-statement, as what is really done is treating only one of them (W), which happens to coincide with another (WM) for some cases. To not be misunderstood, let me stress that I think this is already a significant achievement, just that the authors promise more than they actually provide. }

We think that this is a good idea; we have added a cautionary comment to the introduction. We have also slightly weakened the claim in Section 6.3 about which types we focus on.

***jasper kijkt hier nog even naar

{\bf Is the concept of separately specified rows connected, or even equivalent in some way to the concept of separately specified credal networks? Could a remark/short note be added about it? }

{\color{green}
Yes, our notion of separately specified rows can rather directly be understood as the continuous-time analogue of the separately specified local models that are often adopted in credal networks. We now explain this in a footnote at the end of Definition 24.
}

{\bf Similarly, is the fact that W and WM do not coincide together for multiple time-points  due to the same reasons that make the IDM predictive model lose its nature of linear-vacuous mixture once one consider the two future draws? Or more simply to a difference between (conditional) epistemic irrelevance and strong independence? Could some intuition be given in terms of such concepts? }

{\color{green}
The reason why they do not coincide is indeed, essentially, due to the difference between epistemic irrelevance and strong independence, be it in a continuous-time setting. Following your suggestion, we now provide some intuition on this in a footnote (which is situated a couple of lines before Example 22).
}

{\bf Given the length of the paper, I think various things could be done to make the reading easier: the first is that a list of notations could be provided at the beginning of the paper, especially for those notations that become absent from the paper for a while and then reappear; the second is that it would be useful, in the introduction, to already points out the different parts of the paper, explaining that the first part is quite theoretical, that practitioners can mostly start at section 5 and then go on, and that the case of sensitivity analysis around (homogeneous) Markov chains will not be treated per-se, but rather that computable lower bounds of such sensitivity analysis will be provided; }

We have incorporated this by adding a ``Finding Your Way Around This Paper'' section to the introduction.

**** tabel splitsen (thomas)

**** tekst checken (jasper)

{\bf It would be useful to repeat the proposition/lemma statements in the appendices before providing their proofs, as I found tedious to return to the paper to check those statements. Given the length of the paper, adding these two/three pages may make it more reader friendly, even if it already is quite readable. }

{\color{green}
We have implemented this suggestion.
}

{\bf Concerning the case of sensitivity analysis around Markov chains, I was wondering to which extent does interval-analysis methods tailored for (ordinary) differential equations cannot be used to perform them efficiently, at least for some kind of inferences? I appreciate the pointers to references [17] and [29], but could authors position themselves with respect to works performed by Nacim Ramadan (e.g., �Set membership state and parameter estimation for systems described by nonlinear differential equations�, �Computing reachable sets for uncertain nonlinear monotone systems�, �Guaranteed Parameter Set Estimation for Monotone Dynamical Systems Using Hybrid Automata�) and how this latter could be used in this framework? }

To the best of our understanding, these three references deal with deterministic non-linear systems whose dynamics are described by a finite number of parameters, whose state is described by a real vector, and which can be observed up to some fixed deterministic error. The aim is to use the observations to provide a set-valued estimate for the parameter vector, based on the observations.

These methods are superficially related to ours, in the sense that they consider a set of parameter vectors, but there does not seem to be a mathematical connection that can be used to perform our computations. First of all, these references deal with observations, which we do not. Secondly, even if we were to include observations (and consider a hidden imprecise continuous-time Markov chain), then still, there are a number of crucial differences. For example, their observations are real-valued, whereas ours are categorical. Also, since they consider a deterministic system, there is no need to estimate the parameters; it is just a matter of being able to compute them (or rather the bounds on them). Finally, in our case, at least in the case of sensitivity analysis, the system for which we have set-valued parameters is linear, whereas theirs is non-linear.

That said, it might indeed be true that interval-analysis methods tailored for ordinary differential equations could be used to perform some of our inferences. In particular, it might be possible to use techniques from differential inclusions to perform inferences for $\mathbb{P}_{\mathcal{Q},\mathcal{M}}^{\mathrm{WM}}$, provided that these inferences are about a single time point. However, it seems to us that this would not be very useful, especially since this type of inference is also the solution of a single non-linear differential equation that, as we have shown, is quite easy to compute numerically.

{\bf  I am ok with the current title, but I think it is a bit misleading, as a big part of the paper is not devoted to Markov chains. I think a title such as �Imprecise Time-continuous stochastic processes, and their relations to Markov chains� would do more justice to the paper. Indeed, we may even argue that most of the results given in the paper do not concern Markov chains. }

{\color{green}
While we understand this concern, we have nevertheless decided to keep the current title. Our reasons for doing so are the following. First, despite this paper not yet being accepted for publication, it has already been referenced by several papers that have been published or accepted for publication; we would like for those references to remain unambiguous.

Second, the ``Markov'' in the title refers to the fact that the model satisfies an ``imprecise Markov property'', as we show in the paper. The use of terminology here is consistent with the existing body of literature on (discrete-time) imprecise Markov chains.  (Admittedly, perhaps it would have historically been a good idea to hyphenate the term, since ``imprecise-Markov'' makes it more explicit how the term should be read.)
}

%In light of this second point, however, we have added some additional explanation of how the term should be read to the abstract. This hopefully removes the ambiguity for readers who are less familiar with the subject.

{\bf P17, $\mathcal{P}$ is used here to denote the power set, and is later used to denote sets of processes. Maybe change the notation for the power set?}

{\color{green}
Thanks for spotting this inconsistency. We have changed our notation for the set of all events to $\mathcal{E}(\Omega)$.
}

{\bf P25: the following example established -$>$ the following example establishes}

{\color{green}
Thanks for spotting this typo. It has been fixed.
}

{\bf P34: middle, after ``new transition matrix system'', $t$ should be $r$.}

{\color{green}
Thanks for spotting this typo. It has been fixed.
}

{\bf P49, after equation (59): one of the two $\underline{\mathbb{E}}^{\mathrm{W}}$ should be $\underline{\mathbb{E}}^{\mathrm{WM}}$}

{\color{green}
Thanks for spotting this typo. It has been fixed.
}

{\bf P51: especially if want them -$>$ especially if we want them}

{\color{green}
Thanks for spotting this typo. It has been fixed.
}

{\bf P51: since $\mathcal{Q}$ can very complicated -$>$ since $\mathcal{Q}$ can be very complicated}

{\color{green}
Thanks for spotting this typo. It has been fixed.
}

{\bf P54: and further on: when providing a (numerical) result of the kind $x\pm\epsilon$, I would replace $=$ by $\in$, since the lower probability is a precise value, and the provided answer is an interval induced by error bounds}

*** thomas

*** voetnoot toevoegen bij eerste gebruik, we schrijven plusmin om te bedoelen **inclusie**, en analoog voor functies doen we dat puntsgewijs

*** voor reviewer: omdat we de notatie ook voor functies gebruiken vinden we deze notatie iets handiger, maar we hebben een voetnoot toegevoegd om het te verduidelijken

{\bf P58: equation (69): for all $x_1,x_2,x_3$ should be replaced by $x_0,x_1,x_2$}

{\color{green}
Thanks for spotting this typo. It has been fixed.
}

{\bf P68: lie at the hart -$>$ lie at the heart}

{\color{green}
Thanks for spotting this typo. It has been fixed.
}

\newpage
{\bf Reviewer 3}


{\bf\ldots My obvious criticism involves the length of the manuscript. There are two critical issues in having such a long manuscript on such a theoretical matter. First, it places a heavy toll on reviewers. In fact, (and given that I have not the necessary mathematical background) I have not carefully checked proofs. Second, it lessens the possible impact of the paper: It is very likely that many interesting results will go unnoticed due to the amount of proofs and concepts introduced. This criticism would be of less importance if the length of the paper was mostly due to an effort to be self-contained. This is not the case here: the manuscript clearly contains (at least) two parts that could be presented separately without much loss. In fact, the authors themselves suggest in the introduction that the manuscript consists of 3 parts: one that develops a theory of continuous-time discrete space stochastic process using full conditional probabilities, one that extends that theory to the imprecise case, and one that deals with lower transition operators and their use in characterizing Markovian processes and computing inferences. 

Thus, I recommend to the authors to split their work in two publications. For this work, it should be sufficient to limit the manuscript to the first six sections (and their corresponding proofs in the appendices). This would make the manuscript focused on the formalization and characterization of continuous-time discrete space stochastic process and their imprecise version, an already valuable effort. Note that performing such a cut would already produce a manuscript with over 60 pages. Leaving the rest for another work has another advantage: it would be possible to present the theoretical results that are more tailored for inference and the algorithm together with some (even if unrealistic) experiments, showing that useful (i.e., non vacuous) conclusions can actually be taken from imprecise models (especially when these models contain nonhomegenous and nonMarkovian chains).
}

{\color{green}
We have seriously considered executing such a split---and in fact, had already considered it before---but have, nevertheless, decided not to do so. The reason is the following.

For the main text, we agree with the reviewer that such a split would be feasible, useful and rather natural. However, unfortunately, this is not the case for the proofs of our result. In particular, the proofs of many of the results in the last part of the paper depend heavily on the machinery and results from the first part. Without it, understanding and/or following these proofs would be impossible. Hence, if the last part of this paper would become a second paper, much of the material in the first part would need to be included in that second paper as well. As such, this second paper would be very similar to the current one. For this reason, we prefer to tell the complete theoretical story in a single paper. Applied follow-up papers can then avoid the mathematical machinery, by providing an intuitive explanation and referring to the current papers for details, and can focus on the more practical and computational aspects of our algorithms.
}



Regarding experiments to provide a ``practical proof-of-concept'', we would like to refer to the paper ``Modelling spectrum assignment in a two-service flexi-grid optical link with imprecise continuous-time Markov chains'' by Rottondi \emph{et al.}, for a successful application of the model. The same reference has been added to the paper.

*** jasper: voegt die ref ook bij voorbeeld toe

{\bf Another form of reducing the length is to make it less didactic. Many examples have the intention of showing the existence of objects with certain properties. This is the case, for instance, of Examples 3, 4 and 5. There is no need of presenting the proofs in those examples (this is not the case when the intent of the example is to show the usage of some result). 
The casual reader can certainly take the word of the authors that such examples hold true, while the more interested reader can either obtain these results independently or be pointed to some technical report. }

{\color{green}
We have implemented this suggestion. Examples 3,4,5,10,11,12,16 and 17 have been shortened by removing their proof (which we will now make available in the appendix of a technical report version of the paper).
}

\textcolor{blue}{*** MOET NOG AANGEPAST WORDEN IN DE IJAR VERSIE! ***}

{\bf There is also no need of including the gambling interpretation as an appendix, as this is a rather straightforward concept that is present in many textbooks. }

{\color{green}
We have removed this part of the appendix. The main text now provides pointers to a technical report version of this paper (which does include this appendix), as well as to several references on coherence. 
}

\textcolor{blue}{*** MOET NOG AANGEPAST WORDEN IN DE IJAR VERSIE! (zowel in Sectie 4.1, in Sectie 7.1 als in de appendix) ***}

{\bf page 3, first equation: the subindices $r$ and $t$ appear swapped on the left.}

{\color{green}
Thanks for spotting this typo. It has been fixed.
}

{\bf same page, last paragraph: it is not precisely the case that the current approach does not introduce additional parameters; to make this discussion succinct, note that an imprecise probability model of a binary variable requires at least 2 values while the corresponding precise model requires only one. It is better to say that the proposed approach requires only slightly more parameters (that also depends on how models are specified, a topic that is not given much attention: e.g.,  if a convex set is specified by its vertices, the number of required parameters might be huge) }

{\color{green}
We have implemented this suggestion. Instead of stating that we do not introduce new parameters (which, strictly speaking, is indeed not true), we now state that we do not introduce a large number op new parameters.}


{\bf Page 7, first paragraph of section 2.1: this definition of sequence is rather unusual and in conflict with common use. It is better to define a sequence as a set endowed with the usual total ordering over reals (a sequence usually admits repetitions, which might be created by concatenation, the most common operation on sequences). }

{\color{green}
We do not understand this comment. It seems to us that our definition is equivalent to a (finite) set endowed with the usual ordering, with our restriction to ``non-degenerate'' sequences removing the possibility of repetitions. Also, as explained in the text, we do not concatenate sequences, but consider their union.
}

{\bf Same page and section, last paragraph: you define $[t,s]$ with $t \leq s$, but then assume that $t=t_0 < t_n = s$. Either make the former strict (hence disallowing empty sequences) or allow non strict inequality in the latter }

{\color{green}
It seems to us that the current definition is correct. In fact, it was purposefully phrased like this to cover some edge-cases. Basically, if $t=s$, then $n=0$, and the strict inequalities in the statement $t=t_0<t_1<\dots<t_n=s$ then reduces to $t=t_0=s$, which is not in conflict with $t\leq s$. Note furthemore that we also call the singleton sequence $u=t$ a sequence, and that this is the unique element of the set $\mathcal{U}_{[t,t]}$.
}

{\bf Page 11, Proposition 3.5: I find the notation $\inf\{ \cdot \} > -\infty$ too informal. Better write that there is a constant $c$ such that $\inf\{\cdot\} > c$. }

{\color{green}
The notation is admittedly somewhat informal, but we personally find it useful and intuitive. To alleviate your concern and prevent any ambiguity, we have added a comment in Section 2, explaining how the notation should be read.
}

{\bf same page: due the property -$>$ due to the property}

{\color{green}
Thanks for spotting this typo. It has been fixed.
}

{\bf Page 14, equation (11): some of the inequalities must be strict, otherwise both cases intersect at different points, e.g. $[4/3]\cdot 2^{i+1}$ and $[1/3]\cdot 2^{-i}$  (in fact, it is a bit weird that they intersect at different points) }

{\color{green}
We do not understand where the first of your two expressions comes from. For even $n$, we have that $\varphi_1(\delta_n)=\frac{1}{3}2^{-n}$, whereas for odd $n$, we have that $\varphi_1(\delta_n)=\frac{2}{3}2^{-n}$. In both cases, this value is correctly given by each of the two expression in Equation (11) (for appropriate choices of $i$). We have purposefully chosen to use non-strict inequalities, to make it explicit that the function $\varphi_1(t)$ is continuous. 
}

{\bf Page 15, first paragraph: to be unique one of the sides of the inequality $\delta_(j+1) \leq t \leq \delta_j$ must be strict. }

{\color{green}
Following your earlier suggestion, we have removed this proof from the paper and moved it to the appendix of a technical report version. That said, this was indeed incorrect. Thanks for spotting it. We have fixed it---in the technical report---by making the first inequality strict. 
}

{\bf Same page, first equation: I could not follow the passage to the 2nd equality; please elaborate }

{\color{green}
Following your earlier suggestion, we have removed this proof from the paper and moved it to the appendix of a technical report version. That said, the reason why this passage holds is the following:

If $j$ is odd, it follows from Equations~(11) and (12) that
\begin{equation*}
\varphi_1(t)=t-\nicefrac{2}{3}\delta_{j+1}=(t-\delta_{j+1})+\nicefrac{1}{3}\delta_{j+1}=(t-\delta_{j+1})+\varphi_1(\delta_{j+1})
\end{equation*}
and $\varphi_2(t)=\nicefrac{2}{3}\delta_{j+1}=\varphi_2(\delta_{j+1})$. Similarly, if $j$ is even, it follows that $\varphi_1(t)=\varphi_1(\delta_{j+1})$ and $\varphi_2(t)=\varphi_2(\delta_{j+1})+(t-\delta_{j+1})$. Hence, in both cases, we have that
\begin{equation*}
Q_1\varphi_1(t)+Q_2\varphi_2(t)=Q_1\varphi_1(\delta_{j+1})+Q_2\varphi_2(\delta_{j+1})+Q_j(t-\delta_{j+1}).
\end{equation*}
We have added this argument to the proof---in the technical report.
}

{\bf page 68: second and third inequality -$>$ second and third inequalities}

{\color{green}
Following your earlier suggestion, we have removed this part of the paper---a technical proof in one of the examples---and moved it to the appendix of a technical report version. That said, we have fixed this typo in the technical report.
}

{\bf Page 36: the discussion in the last paragraph is too vague (especially at this point). To comment on computational complexity one need first to formalize how these models are described (so that we known in what terms are we measuring complexity). While the representation of a homogenous imprecise MC is more or less standard (e.g., a set of inequalities or a set of vertices), it is not clear what would be a convenient representation for non homogenous iMCs. It is certainly not proved that computing lower expectations with homogenous MCs is harder: this is at best a conjecture. }

We have weakened the statement to make it more explicit that this is a conjecture. 

***moet nog

*** thomas


%\textcolor{red}{*** MOET NOG GEBEUREN! ***}

{\bf For instance if a homogenous iMC is represented by a finite list of vertices, that a lower expectation can be obtained by computing the corresponding expectation at each vertex, and this is linear in input. Note that this is a lower bound on complexity of other types. }

{\color{green}
We do not understand this comment. In fact, our Example 15 demonstrates that one \emph{cannot} (in general) compute the lower expectation by computing the precise expectation at the vertices of the parameter set, because the optimum is---in this example---obtained at an interior point of the parameter set.
}


{\bf page 38: to obtained for -$>$ to obtain \emph{or} to be obtained}

{\color{green}
Thanks for spotting this typo. It has been fixed.
}

{\bf Same page, 2 sentences below: I do not understand what the authors mean by computing by brute force as the number of solutions is in principle infinite }

{\color{green}
The term ``brute force'' was here used to mean an exhaustive search through the (very finely) discretised parameter-space, so as to find a numerical approximation of the lower expectation and an idea about the range of precise expectations that can be obtained within the set. We have correspondingly changed the wording in the paper.}

*** thomas

\textcolor{red}{*** MOET NOG GEBEUREN! ***}


{\bf Same page, 2 paragraphs below:  ``the computational complexity will in general explode''. What does that mean? Perhaps you mean that the size of the representation will increase exponentially? In any case, better not make such claims about complexity (there is an unfounded claim about intractability in the following sentence) }

*** thomas
%\textcolor{red}{*** MOET NOG GEBEUREN! ***}

{\bf page 68: hart -$>$ heart}

{\color{green}
Thanks for spotting this typo. It has been fixed.
}

{\bf References: many names need to be capitalized: Markov, Bayes, Bayesian; It should be H1N1PDm in Ref. [25]}

{\color{green}
Thanks for spotting these typos. They have all been fixed.
}

\end{document}
