\documentclass[twoside,11pt]{letter}
\usepackage{amsmath}

\begin{document}

Dear Sir/Madam,

We would very much like to thank you for the time and effort that you took in reviewing our paper ``Imprecise Continuous-Time Markov Chains'' for the International Journal of Approximate Reasoning.

{\bf ******* En dan wel nog wat meer dank }

With kind regards,

Thomas Krak\\
Jasper De Bock\\
Arno Siebes
\newline\newline

\newpage
{\bf Reviewer 1}

\ldots~\newline
{\bf 
This justifies the length of the manuscript, although, in my opinion at some parts it could be shortened 
by omitting some straightforward proofs or referring to the existing theory of 
precise continuous time Markov chains, especially in Sections 3 and 5. }

We agree with the observation that some of the proofs/results in Section 5 are conceptually similar to known results from the literature. However, we are not aware of other work that uses full conditional probabilities and/or coherence to characterise (precise) continuous-time Markov chains. Furthermore, the mathematical equivalence between our characterisation and, say, the measure-theoretic one, does not seem straightforward to us (or even, indeed, \emph{exactly} true, since we do not require e.g. $\sigma$-additivity). This makes it difficult to refer back to known results. We have therefore decided to retain the proofs from Section 5.

Regarding the more straightforward proofs from Section 3, we agree that these could in principle be removed. However, those would then be the only (formal) statements in the paper that are given without a corresponding proof in the appendix, which we feel could lead to confusion. Since the proofs in question are only several lines long, adding a ``left-as-an-exercise'' comment to the main text would also not help (much) to reduce the overall length of the paper. We have therefore kept these proofs in the appendix for now, but are willing to remove them should you insist.

{\bf The authors claim that equation (57) is stronger than equation (56). 
This is clearly true if the operators would be defined on infinitely dimensional spaces, 
but in the finite dimensional space I do not see why these equations would not be equivalent, 
especially considering similar equivalence in Lemma F2. If this is in fact so, this claim should be additionally justified. }

****** DEZE MOET IK NOG DOEN

{\bf Lemma G3: In the proof, third line, it is suggested that the norm of Q(f) is smaller than the maximal norm of a 
vector that bounds it from above or below. This might be true in this case, but it does not just follow from such an inequality. 
Consider, for instance, $(-1, 0) \leq (-1, 1) \leq (0, 1)$, where the norm of the middle vector is greater than the norm of both bounding vectors. }

The implication in the third line (i.e. that Equation (132) holds) follows from the fact that we use the \emph{maximum} norm, rather than e.g. the Euclidean norm.

Similarly, in the example above, we have under the maximum norm that
\begin{equation}
\lVert(-1,0)\rVert = \lVert(-1,1)\rVert = \lVert(0,1)\rVert = 1\,,
\end{equation}
and hence also $\lVert(-1,1)\rVert \leq \max\{\lVert(-1,0)\rVert, \lVert(0,1)\rVert\}$, as per the statement.

{\bf Lemma G.4 could be replaced by an easy observation.   }

We unfortunately do not really see an easier argument than the one that was given; note that we cannot directly apply Property LT5 to all $n$ steps simultaneously, nor is Equation (49) sufficient.

Regarding the direct use of Equation (49) to try to construct an easier argument, note that for functions $f \leq g$ it does not in general hold that also $\underline{Q}f \leq \underline{Q}g$. For example, consider $(-1, 0) \leq (0, 0)$, and the transition rate matrix $Q$ (which is trivially a lower transition rate operator $\underline{Q}$), given by
\begin{equation}
Q = \left[\begin{array}{rr}
-1 & 1 \\
0 & 0
\end{array}\right]
\end{equation}
for which it holds that
\begin{equation}
Q\left[\begin{array}{r}-1 \\ 0\end{array}\right] = \left[\begin{array}{r}
1 \\ 0\end{array}\right] > Q\left[\begin{array}{r}0 \\ 0\end{array}\right] = \left[\begin{array}{r}0 \\ 0\end{array}\right]\,.
\end{equation}

In any case, the induction argument that we give in Lemma G.4 is the simplest that we have been able to find. If you can provide an easier argument we would be more than willing to incorporate that instead.

{\bf The notion of well behavedness, as far as I understand, means that the corresponding transition systems satisfy the Lipschitz condition. 
This condition is very often applied in the theory of dynamic systems. 
The authors should check whether any implications of well behavedness coincide with known results based on Lipschitz condition.  }

We are actually not entirely sure if well-behavedness and Lipschitz-continuity coincide exactly, although it is clear that the latter implies the former.

**** DE ANDERE KANT OP IS NIET HELEMAAL TRIVIAAL. TEN EERSTE, WE EISEN HET ALLEEN OP LOKAAL NIVEAU, TEN TWEEDE EISEN WE NIET DAT DE ``LIPSCHITZ CONSTANTE'' ONAFHANKELIJK IS VAN $t$. 

*** Dat wil zeggen, de well-behavedness-eis kan je herschrijven als dat, voor alle $t$, er een $c_t\in R$ is zodanig dat, voor $\Delta$ klein genoeg,
\begin{equation}
\lvert P(X_{t+\Delta}\,\vert\,X_t) - P(X_{t}\,\vert\,X_t) \rvert < \Delta c_t
\end{equation}
Het eerste punt is dus dat we niks eisen voor `grote' $\Delta$. Het tweede punt is dat om Lipschitz-continue te zijn, we ook zouden moeten hebben dat
\begin{equation*}
\sup_{t} c_t < +\infty\,.
\end{equation*}
Of dat geimplicteerd is (samen met de definities van stochastische processen).... ik heb echt geen idee.

{\bf Why are some environments, e.g. examples definitions, enumerated differently than another? }

To answer the question, this was due to a stylistic preference of one of the authors. Regardless, the current version has environments enumerated as per the IJAR/Elsevier style.

\end{document}
