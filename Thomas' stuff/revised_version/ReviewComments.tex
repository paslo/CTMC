\documentclass[twoside,11pt]{letter}
\usepackage{amsmath}

\begin{document}

Dear Sir/Madam,

We would very much like to thank you for the time and effort that you took in reviewing our paper ``Imprecise Continuous-Time Markov Chains'' for the International Journal of Approximate Reasoning.

{\bf ******* En dan wel nog wat meer dank }

With kind regards,

Thomas Krak\\
Jasper De Bock\\
Arno Siebes
\newline\newline

\newpage
{\bf Reviewer 1}

\ldots~\newline
{\bf 
This justifies the length of the manuscript, although, in my opinion at some parts it could be shortened 
by omitting some straightforward proofs or referring to the existing theory of 
precise continuous time Markov chains, especially in Sections 3 and 5. }

We agree with the observation that some of the proofs/results in Section 5 are conceptually similar to known results from the literature. However, we are not aware of other work that uses full conditional probabilities and/or coherence to characterise (precise) continuous-time Markov chains. Furthermore, the mathematical equivalence between our characterisation and, say, the measure-theoretic one, does not seem straightforward to us (or even, indeed, \emph{exactly} true, since we do not require e.g. $\sigma$-additivity). This makes it difficult to refer back to known results. We have therefore decided to retain the proofs from Section 5.

Regarding the more straightforward proofs from Section 3, we agree that these could in principle be removed. However, those would then be the only (formal) statements in the paper that are given without a corresponding proof in the appendix, which we feel could lead to confusion. Since the proofs in question are only several lines long, adding a ``left-as-an-exercise'' comment to the main text would also not help (much) to reduce the overall length of the paper. We have therefore kept these proofs in the appendix for now, but are willing to remove them should you insist.

{\bf The authors claim that equation (57) is stronger than equation (56). 
This is clearly true if the operators would be defined on infinitely dimensional spaces, 
but in the finite dimensional space I do not see why these equations would not be equivalent, 
especially considering similar equivalence in Lemma F2. If this is in fact so, this claim should be additionally justified. }

****** DEZE MOET IK NOG DOEN

{\bf Lemma G3: In the proof, third line, it is suggested that the norm of Q(f) is smaller than the maximal norm of a 
vector that bounds it from above or below. This might be true in this case, but it does not just follow from such an inequality. 
Consider, for instance, $(-1, 0) \leq (-1, 1) \leq (0, 1)$, where the norm of the middle vector is greater than the norm of both bounding vectors. }

The implication in the third line (i.e. that Equation (132) holds) follows from the fact that we use the \emph{maximum} norm, rather than e.g. the Euclidean norm.

Similarly, in the example above, we have under the maximum norm that
\begin{equation}
\lVert(-1,0)\rVert = \lVert(-1,1)\rVert = \lVert(0,1)\rVert = 1\,,
\end{equation}
and hence also $\lVert(-1,1)\rVert \leq \max\{\lVert(-1,0)\rVert, \lVert(0,1)\rVert\}$, as per the statement.

{\bf Lemma G.4 could be replaced by an easy observation.   }

We unfortunately do not really see an easier argument than the one that was given; note that we cannot directly apply Property LT5 to all $n$ steps simultaneously, nor is Equation (49) sufficient.

Regarding the direct use of Equation (49) to try to construct an easier argument, note that for functions $f \leq g$ it does not in general hold that also $\underline{Q}f \leq \underline{Q}g$. For example, consider $(-1, 0) \leq (0, 0)$, and the transition rate matrix $Q$ (which is trivially a lower transition rate operator $\underline{Q}$), given by
\begin{equation}
Q = \left[\begin{array}{rr}
-1 & 1 \\
0 & 0
\end{array}\right]
\end{equation}
for which it holds that
\begin{equation}
Q\left[\begin{array}{r}-1 \\ 0\end{array}\right] = \left[\begin{array}{r}
1 \\ 0\end{array}\right] > Q\left[\begin{array}{r}0 \\ 0\end{array}\right] = \left[\begin{array}{r}0 \\ 0\end{array}\right]\,.
\end{equation}

In any case, the induction argument that we give in Lemma G.4 is the simplest that we have been able to find. If you can provide an easier argument we would be more than willing to incorporate that instead.

{\bf The notion of well behavedness, as far as I understand, means that the corresponding transition systems satisfy the Lipschitz condition. 
This condition is very often applied in the theory of dynamic systems. 
The authors should check whether any implications of well behavedness coincide with known results based on Lipschitz condition.  }

We are actually not entirely sure if well-behavedness and Lipschitz-continuity coincide exactly, although it is clear that the latter implies the former.

**** DE ANDERE KANT OP IS NIET HELEMAAL TRIVIAAL. TEN EERSTE, WE EISEN HET ALLEEN OP LOKAAL NIVEAU, TEN TWEEDE EISEN WE NIET DAT DE ``LIPSCHITZ CONSTANTE'' ONAFHANKELIJK IS VAN $t$. 

*** Dat wil zeggen, de well-behavedness-eis kan je herschrijven als dat, voor alle $t$, er een $c_t\in R$ is zodanig dat, voor $\Delta$ klein genoeg,
\begin{equation}
\lvert P(X_{t+\Delta}\,\vert\,X_t) - P(X_{t}\,\vert\,X_t) \rvert < \Delta c_t
\end{equation}
Het eerste punt is dus dat we niks eisen voor `grote' $\Delta$. Het tweede punt is dat om Lipschitz-continue te zijn, we ook zouden moeten hebben dat
\begin{equation*}
\sup_{t} c_t < +\infty\,.
\end{equation*}
Of dat geimplicteerd is (samen met de definities van stochastische processen).... ik heb echt geen idee.

{\bf Why are some environments, e.g. examples definitions, enumerated differently than another? }

To answer the question, this was due to a stylistic preference of one of the authors. Regardless, the current version has environments enumerated as per the IJAR/Elsevier style.

\newpage
{\bf Reviewer 2}

\ldots~\newline
{\bf Could some words be said about the case where different sets may be given for different time intervals (e.g., modelling the fact that, as time goes along, we are less and less certain what is our transition models)? Does it modify significantly the authors result, or do we have to simply apply the provided methods for each time-intervals with different models? Indeed, page 32, at the top, suddenly sets of rate matrices become time-independent, whereas the page before (equation 38), rate matrices were indeed time-dependent. I understand very well that rate matrices may be changing within the set of rate matrices, hence that equation (38) can somehow be included in sets of rate matrices, yet one wonder what would happen if sets of rate matrices were themselves time-dependent. }

This is indeed a very interesting question, and something that might be worth looking into in future work. The obvious subtlety is then the kind of continuity assumptions that should be imposed on these changing parameter-sets.

We will not add any in-depth research on this to the present paper, but have added some discussion and remarks to Sections XX and YY.

**** DAT MOET NOG WEL GEBEUREN

{\bf Similarly, the third part exclusively deals with generic stochastic process that are within the set of provided transition matrices. Some warnings should be sent to the reader about that (in particular practitioners), as well as some proper cautionary notes: while the given bounds certainly provides cautious bounds for the other cases, example 15 certainly suggests that these bounds may be quite conservative, and I guess the same may happen between W and WM in the case of multiple time points. In particular, stating at the end of P35 that �one of the main aim of this paper � for the different types of ICTMC�s �� seems to me quite an over-statement, as what is really done is treating only one of them (W), which happens to coincide with another (WM) for some cases. To not be misunderstood, let me stress that I think this is already a significant achievement, just that authors promise more than they actually provide. }

We think that this is a good idea; we have added some cautionary comments in Sections XX and YY. We have also slightly weakened the claim about which types we focus on.

{\bf Is the concept of separately specified rows connected, or even equivalent in some way to the concept of separately specified credal networks? Could a remark/short note be added about it? }

Yes, it can rather directly be understood as the continuous-time analogue to separately specified credal networks. We have added a footnote remarking on this to Section XX.

**** DIT MOET NOG WEL GEBEUREN

{\bf Similarly, is the fact that W and WM do not coincide together for multiple time-points  due to the same reasons that make the IDM predictive model lose its nature of linear-vacuous mixture once one consider the two future draws? Or more simply to a difference between (conditional) epistemic irrelevance and strong independence? Could some intuition be given in terms of such concepts? }

*** het tweede, we voegen een voetnoot toe

**** JASPER, DEZE LAAT IK AAN JOU OVER :)

{\bf Given the length of the paper, I think various things could be done to make the reading easier: the first is that a list of notations could be provided at the beginning of the paper, especially for those notations that become absent from the paper for a while and then reappear; the second is that it would be useful, in the introduction, to already points out the different parts of the paper, explaining that the first part is quite theoretical, that practitioners can mostly start at section 5 and then go on, and that the case of sensitivity analysis around (homogeneous) Markov chains will not be treated per-se, but rather that computable lower bounds of such sensitivity analysis will be provided; }

We have incorporated this by adding a ``Finding Your Way Around This Paper'' section to the introduction.

**** DAT IS NOG WEL EEN WORK IN PROGRESS

{\bf It would be useful to repeat the proposition/lemma statements in the appendices before providing their proofs, as I found tedious to return to the paper to check those statements. Given the length of the paper, adding these two/three pages may make it more reader friendly, even if it already is quite readable. }

These have been added.

{\bf Concerning the case of sensitivity analysis around Markov chains, I was wondering to which extent does interval-analysis methods tailored for (ordinary) differential equations cannot be used to perform them efficiently, at least for some kind of inferences? I appreciate the pointers to references [17] and [29], but could authors position themselves with respect to works performed by Nacim Ramadan (e.g., �Set membership state and parameter estimation for systems described by nonlinear differential equations�, �Computing reachable sets for uncertain nonlinear monotone systems�, �Guaranteed Parameter Set Estimation for Monotone Dynamical Systems Using Hybrid Automata�) and how this latter could be used in this framework? }

**** dit is.. niet helemaal triviaal

{\bf  I am ok with the current title, but I think it is a bit misleading, as a big part of the paper is not devoted to Markov chains. I think a title such as �Imprecise Time-continuous stochastic processes, and their relations to Markov chains� would do more justice to the paper. Indeed, we may even argue that most of the results given in the paper do not concern Markov chains. }

*** we houden deze titel, ook gezien de hoeveelheid referenties die het inmiddels heeft

*** we zullen wel ook de "impreciese markov eigenschap" noemen in de abstract

{\bf *** Typos and notation } 

*** Thanks, these are all fixed now.

**** MOET NOG WEL GEBEUREN

\newpage
{\bf Reviewer 3}

\ldots~\newline
{\bf My obvious criticism involves the length of the manuscript. There are two critical issues in having such a long manuscript on such a theoretical matter. First, it places a heavy toll on reviewers. In fact, (and given that I have not the necessary mathematical background) I have not carefully checked proofs. Second, it lessens the possible impact of the paper: It is very likely that many interesting results will go unnoticed due to the amount of proofs and concepts introduced. }

**** We zijn ons ervan bewust, waarvoor zeer veel dank

{\bf This criticism would be of less importance if the length of the paper was mostly due to an effort to be self-contained. This is not the case here: the manuscript clearly contains (at least) two parts that could be presented separately without much loss. In fact, the authors themselves suggest in the introduction that the manuscript consists of 3 parts: one that develops a theory of continuous-time discrete space stochastic process using full conditional probabilities, one that extends that theory to the imprecise case, and one that deals with lower transition operators and their use in characterizing Markovian processes and computing inferences. 

Thus, I recommend to the authors to split their work in two publications. }

*** We hebben hier over gedacht, maar besloten om het niet te doen. Ondanks de conceptuele scheiding zijn er ook (te) veel dependencies waardoor het tweede deel te veel zou moeten overnemen van het eerste deel, wat het nut van opsplitsen nogal teniet doet.

{\bf Leaving the rest for another work has another advantage: it would be possible to present the theoretical results that are more tailored for inference and the algorithm together with some (even if unrealistic) experiments, showing that useful (i.e., non vacuous) conclusions can actually be taken from imprecise models (especially when these models contain nonhomegenous and nonMarkovian chains). }

**** Wat betreft experimenten, zie telecom paper

{\bf Another form of reducing the length is to make it less didactic. Many examples have the intention of showing the existence of objects with certain properties. This is the case, for instance, of Examples 3, 4 and 5. There is no need of presenting the proofs in those examples (this is not the case when the intent of the example is to show the usage of some result). 
The casual reader can certainly take the word of the authors that such examples hold true, while the more interested reader can either obtain these results independently or be pointed to some technical report. }

**** We agree that this is a valid option to reduce the length of the paper. We have replaced the proofs with simpler claims in Examples X, Y, Z

{\bf There is also no need of including the gambling interpretation as an appendix, as this is a rather straightforward concept that is present in many textbooks. }

**** Prima, we verwijzen in plaats daarvan naar REFERENCE


{\bf Page 3, last paragraph: it is not precisely the case that the current approach does not introduces additional parameters; to make this discussion succinct, note that an imprecise probability model of a binary variable requires at least 2 values while the corresponding precise model requires only one. It is better to say that the proposed approach requires only slightly more parameters (that also depends on how models are specified, a topic that is not given much attention: e.g.,  if a convex set is specified by its vertices, the number of required parameters might be huge) }

*** we voegen een comment toe

{\bf Page 7, first paragraph of section 2.1: this definition of sequence is rather unusual and in conflict with common use. It is better to define a sequence as a set endowed with the usual total ordering over reals (a sequence usually admits repetitions, which might be created by concatenation, the most common operation on sequences). }

We regret to say that we do not entirely understand your concern here. Our definition seems equivalent to a (finite) set endowed with the usual ordering, with our restriction to ``non-degenerate'' sequences removing the possibility of repetitions.

{\bf Same page and section, last paragraph: you define $[t,s]$ with $t \leq s$, but then assume that $t=t_0 < t_n = s$. Either make the former strict (hence disallowing empty sequences) or allow non strict inequality in the latter }

The current definition is correct for our present purposes, and is purposefully phrased like this to cover some edge-cases. Note that we also call the singleton sequence $u=t$ a sequence, and that this is the unique element of the set $\mathcal{U}_{[t,t]}$. We could add the qualifier that the inequality $t=t_0<t_n=s$ should only hold for $n\geq 1$, but we worry that this might come across as somewhat pedantic.

{\bf Page 11, Proposition 3.5: I find the notation $\inf\{ \cdot \} > -\infty$ too informal. Better write that there is a constant $c$ such that $\inf\{\cdot\} > c$. }

The notation is admittedly somewhat informal, but we personally find the shorthand useful. To alleviate your concern and prevent any ambiguity, we have added a comment to Section 2 explaining how the notation should be read.

{\bf Page 14, equation (11): some of the inequalities must be strict, otherwise both cases intersect at different points, e.g. $[4/3]\cdot 2^{i+1}$ and $[1/3]\cdot 2^{-i}$  (in fact, it is a bit weird that they intersect at different points) }

***** hoe zat dit nou ook alweer? klopte toch?

{\bf Page 15, first paragraph: to be unique one of the sides of the inequality $\delta_(j+1) \leq t \leq \delta_j$ must be strict. }

***** hoe zat dit nou ook alweer? klopte toch?

{\bf Same page, first equation: I could not follow the passage to the 2nd equality; please elaborate }

**** waarschijnlijk gaat dit bewijs weg, maargoed. moet dan even uitgeschreven worden.

{\bf Page 36: the discussion in the last paragraph is too vague (especially at this point). To comment on computational complexity one need first to formalize how these models are described (so that we known in what terms are we measuring complexity). While the representation of a homogenous imprecise MC is more or less standard (e.g., a set of inequalities or a set of vertices), it is not clear what would be a convenient representation for non homogenous iMCs. It is certainly not proved that computing lower expectations with homogenous MCs is harder: this is at best a conjecture. }

We have weakened the statement to make it more explicit that this is a conjecture. *** dit moet nog gebeuren

{\bf For instance if a homogenous iMC is represented by a finite list of vertices, that a lower expectation can be obtained by computing the corresponding expectation at each vertex, and this is linear in input. Note that this is a lower bound on complexity of other types. }

We regret to say that we do not entirely understand what you mean here. Our Example 15 demonstrates that one \emph{cannot} (in general) compute the lower expectation by computing the precise expectation at the vertices of the parameter set.

{\bf Same page, 2 sentences below: I do not understand what the authors mean by computing by brute force as the number of solutions is in principle infinite }

The term ``brute force'' was here used to mean an exhaustive search through the (very finely) discretised parameter-space, so as to find a numerical estimate of the lower expectation and the range of precise expectations that can be obtained within the set. We have correspondingly changed the wording in the paper.

**** DAT MOET NOG WEL GEBEUREN

{\bf Same page, 2 paragraphs below:  "the computational complexity will in general explode". What does that mean? Perhaps you mean that the size of the representation will increase exponentially? In any case, better not make such claims about complexity (there is an unfounded claim about intractability in the following sentence) }

{\bf *** Typos and notation}

**** Thanks, these are all fixed now (MOET NOG WEL GEBEUREN)

\end{document}
