\documentclass[twoside,11pt]{article}
\usepackage{isipta}

%% Comment these two lines if you don't need UTF8
\usepackage[utf8]{inputenc}
\inputencoding{utf8} 

% Any additional packages needed should be included after isipta.sty
% Note that pgm.sty includes epsfig, amssymb, natbib and graphicx,
% and defines many common macros, such as 'proof' and 'example'.
% It also sets the bibliographystyle 

%% Put here the import commands of the packages you need and your custom commands 
\usepackage{hyperref,color,soul,booktabs}
\setulcolor{blue}
\newcommand{\BibTeX}{\textsc{B\kern-0.1emi\kern-0.017emb}\kern-0.15em\TeX}

% Running title and authors 
\ShortHeadings{ISIPTA '17: Author Instructions}{Antonucci et al.} 

%\pgmheading{1}{2000}{1-48}{4/00}{10/00}{Professorson and Teacherman}
%\firstpageno{10}

\begin{document}

% Title and authors
\title{ISIPTA '17: Author Instructions}
\author{\name Alessandro Antonucci \email alessandro@idsia.ch\\
\name Giorgio Corani \email giorgio@idsia.ch\\
\addr Istituto Dalle Molle di Studi Sull'Intelligenza Artificiale (IDSIA)\\
Lugano (Swizterland)
\AND
\name S\'ebastien Destercke \email sebastien.destercke@hds.utc.fr\\
\addr Heuristique et Diagnostic des Systèmes Complexes (HEUDIASYC)\\
Compi\`egne (France)\\
\AND
\name In\'es Couso \email couso@uniovi.es\\
\addr Universidad de Oviedo\\
Gijon (Spain)}
\maketitle
% Abstract and keywords
\begin{abstract}%   <- trailing '%' for backward compatibility of .sty file
The abstract is a mandatory element that should summarize the contents of the paper. It should be confined within a single paragraph.
\end{abstract}
\begin{keywords}
We would like to encourage you to list your keywords here. They should be separated by semicolons and the last is followed by a full stop.
\end{keywords}


\section{Major Guidelines}
Your ISIPTA '17 paper should be submitted on the {\href{https://easychair.org/conferences/?conf=isipta17}{\ul{ISIPTA '17 Easychair page}}. Please check the {\href{http://isipta.idsia.ch}{conference website} for the submission deadlines.
\begin{itemize}
\item Limit your text, bibliography and appendices included, to 12 pages.
\item Following the JMLR naming conventions, assign to your paper the following (lowercase) ID: surname of first author followed by two-digit year of publication and an iterative letter code if multiple papers have the same first author. E.g., the ID of the paper by Alessandro Antonucci should be {\tt antonucci17} or {\tt antonucci17a} in case of multiple papers submitted.
\item Your paper should be typeset using \LaTeX. Put all the files required for the compilation (sources, images, etc.) in a folder. Use the ID for the folder name and for the main \LaTeX file (e.g., {\tt antonucci17.tex}). Compress the folder in a zip or tarball archive before uploading it on Easychair.
\item The very preamble of your {\tt .tex} file should be:
\begin{verbatim}
\documentclass[twoside,11pt]{article}
\usepackage{isipta}
\end{verbatim}
\item You can find the {\tt isipta.sty} file in the author's kit. Alternatively, get it \href{http://www.idsia.ch/~alessandro/isipta.sty}{\ul{here}}. This style file includes the {\tt amssymb}, {\tt natbib} and {\tt graphicx} packages, and defines many common macros, such as {\tt proof} and {\tt example}.  Do not include the package {\tt theorem}. If possible, stick to standard \LaTeX packages. Define your own commands (e.g., {\tt $\backslash$eg} for ``e.g.'') in the main {\tt .tex} file. Don't use additional {\tt .sty} files.
\item For images use PDF or PNG files, included with {\tt $\backslash$includegraphics\{imagefile\}}. Avoid EPS images, the {\tt epsfig} package, and the {\tt $\backslash$graphicspath} command. 
\item References should be managed using \BibTeX. The {\tt bibliographystle} is already specified in the style file. Use the standard {\tt $\backslash$cite} and {\tt $\backslash$citep} for citations. E.g., we can cite a paper like this \citep{einstein}, or we can say that \citet{latexcompanion} wrote a nice book.
\end{itemize}

\section{Copyright Form}
\begin{itemize}
\item If your paper has been accepted and you are preparing the camera-ready version, download the \href{http://jmlr.csail.mit.edu/proceedings/jmlrPublicationForm.pdf}{\ul{copyright form}}. Print, fill, and sign the form. The name of the PDF scan should be the ID with the suffice ``Permission'' (e.g., {\tt antonucci17Permission.pdf}). This file should be included in the folder with the sources to be uploaded on Easychair.
\end{itemize}

\section{Some Other Guidelines}
\begin{itemize}
\item The title of the paper and of the sections should capitalize on the first letter of each word, except for articles, conjunctions, and prepositions. If possible, try to avoid {\tt $\backslash$subsubsections}.
\item Ensure that below each figure and table there is a caption (e.g., see Table~\ref{tab:mytable}).
\item Acknowledgements should go at the end, before appendices and references.
\item For UTF8 encoding (e.g., to quickly write \emph{astéroïde} in your paper), add to the preamble:
\begin{verbatim} 
\usepackage[utf8]{inputenc}
\inputencoding{utf8} 
\end{verbatim}
\item Please do not change the size of the fonts, the line spacing, and the margins.
\end{itemize}

\begin{table}
\centering
\begin{tabular}{llr}
\toprule
First name & Last Name & Grade \\
\midrule
John & Doe & $7.5$ \\
Richard & Miles & $2$ \\
\bottomrule
\end{tabular}
\caption{A (useless) table.}\label{tab:mytable}
\end{table}
\appendix
\acks{We would like to acknowledge support for this project from the Random Science Foundation.}
\section{Proofs}
\label{app:theorem}
In this appendix we present a random filler.
\vskip 0.2in
\bibliography{references}
\end{document}
